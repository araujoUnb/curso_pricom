% ============================================
% SEÇÃO 3.9: COMPUTAÇÃO NUMÉRICA DA TRANSFORMADA DE FOURIER
% ============================================

\subsection{Motivação}

\begin{frame}{Por que Transformada Discreta?}

\textbf{Transformada de Fourier contínua:}

\[
F(\omega) = \int_{-\infty}^{\infty} f(t) e^{-j\omega t} dt
\]

\textbf{Problemas para implementação computacional:}

\begin{enumerate}
\item \textbf{Integral infinita:} Computadores não processam intervalos infinitos
\item \textbf{Tempo contínuo:} Necessário amostrar o sinal discretamente
\item \textbf{Frequência contínua:} Impossível calcular para todo $\omega$
\end{enumerate}

\vspace{0.3cm}

\textbf{Solução:} \textbf{Transformada Discreta de Fourier (DFT)}

\begin{itemize}
\item Opera em sequências finitas de amostras
\item Produz espectro em frequências discretas
\item Computável em tempo finito
\item Base para análise espectral prática
\end{itemize}

\end{frame}

% ============================================

\subsection{Amostragem de Sinais}

\begin{frame}{Teorema da Amostragem (Nyquist-Shannon)}

\textbf{Problema:} Converter sinal contínuo $f(t)$ em amostras discretas $f[n]$.

\begin{block}{Teorema de Nyquist-Shannon}
Um sinal de banda limitada $f(t)$ com frequência máxima $f_{max}$ pode ser perfeitamente reconstruído a partir de suas amostras se a taxa de amostragem $f_s$ satisfizer:
\[
f_s \geq 2f_{max}
\]
\end{block}

\textbf{Definições:}

\begin{itemize}
\item \textbf{Taxa de Nyquist:} $f_N = 2f_{max}$ (taxa mínima)
\item \textbf{Período de amostragem:} $T_s = 1/f_s$
\item \textbf{Amostras:} $f[n] = f(nT_s)$ para $n = 0, 1, 2, ..., N-1$
\end{itemize}

\vspace{0.3cm}

\textbf{Consequência:} Se $f_s < 2f_{max}$, ocorre \textbf{aliasing} (distorção espectral).

\end{frame}

% ============================================

\begin{frame}{Aliasing}

\textbf{Aliasing} ocorre quando a taxa de amostragem é insuficiente.

\textbf{Efeito:}

Frequências acima de $f_s/2$ aparecem como frequências mais baixas (aliases):

\[
f_{alias} = |f_{sinal} - kf_s|
\]

para algum inteiro $k$ que minimiza $f_{alias}$.

\vspace{0.3cm}

\textbf{Exemplo clássico:} Rodas de carroça em filmes aparecem girando para trás.

\textbf{Prevenção:}

\begin{enumerate}
\item \textbf{Filtro anti-aliasing:} Passa-baixas antes da amostragem
   
   Remove componentes acima de $f_s/2$

\item \textbf{Sobreamostragem:} Usar $f_s \gg 2f_{max}$
   
   Margem de segurança, facilita filtragem
\end{enumerate}

\textbf{Prática comum:} $f_s = (2.5 \text{ a } 4) \times 2f_{max}$

\end{frame}

% ============================================

\subsection{Transformada Discreta de Fourier}

\begin{frame}{Definição da DFT}

Dado sequência finita de $N$ amostras: $x[0], x[1], ..., x[N-1]$

\begin{block}{Transformada Discreta de Fourier (DFT)}
\[
X[k] = \sum_{n=0}^{N-1} x[n] e^{-j2\pi kn/N}, \quad k = 0, 1, ..., N-1
\]
\end{block}

\begin{block}{DFT Inversa (IDFT)}
\[
x[n] = \frac{1}{N} \sum_{k=0}^{N-1} X[k] e^{j2\pi kn/N}, \quad n = 0, 1, ..., N-1
\]
\end{block}

\textbf{Parâmetros:}

\begin{itemize}
\item $N$: Número de amostras (comprimento da DFT)
\item $x[n]$: Amostras no domínio do tempo
\item $X[k]$: Coeficientes espectrais (geralmente complexos)
\item $k$: Índice de frequência discreta
\end{itemize}

\end{frame}

% ============================================

\begin{frame}{Interpretação da DFT}

\textbf{Frequências discretas:}

A DFT avalia o espectro em $N$ frequências igualmente espaçadas:

\[
f_k = \frac{k f_s}{N} = \frac{k}{NT_s}, \quad k = 0, 1, ..., N-1
\]

ou em frequência angular:

\[
\omega_k = \frac{2\pi k}{N T_s}
\]

\textbf{Resolução em frequência:}

\[
\Delta f = \frac{f_s}{N} = \frac{1}{T_{obs}}
\]

onde $T_{obs} = NT_s$ é o tempo total de observação.

\vspace{0.3cm}

\textbf{Conclusões importantes:}

\begin{itemize}
\item Maior $N$ $\rightarrow$ melhor resolução em frequência
\item Maior $f_s$ $\rightarrow$ maior faixa de frequências (até $f_s/2$)
\item Compromisso: resolução vs. faixa vs. custo computacional
\end{itemize}

\end{frame}

% ============================================

\begin{frame}{Propriedades da DFT}

\textbf{Periodicidade:}

\[
X[k + N] = X[k], \quad x[n + N] = x[n]
\]

A DFT assume sinais periódicos com período $N$.

\vspace{0.3cm}

\textbf{Simetria (para sinais reais):}

Se $x[n]$ é real, então:

\[
X[N-k] = X^*[k]
\]

Logo, $|X[N-k]| = |X[k]|$ (espectro de magnitude simétrico).

Basta calcular $k = 0, 1, ..., N/2$ (metade dos pontos).

\vspace{0.3cm}

\textbf{Linearidade:}

\[
\text{DFT}\{ax_1[n] + bx_2[n]\} = aX_1[k] + bX_2[k]
\]

\end{frame}

% ============================================

\begin{frame}{Relação DFT-TF Contínua}

\textbf{Aproximação:} A DFT aproxima a Transformada de Fourier contínua.

Para sinal amostrado $f(t)$ com $f[n] = f(nT_s)$:

\[
X[k] \approx T_s F(\omega_k)
\]

onde $\omega_k = 2\pi k/(NT_s)$.

\vspace{0.3cm}

\textbf{Derivação intuitiva:}

A integral da TF é aproximada por uma soma (regra do retângulo):

\[
F(\omega) = \int_{-\infty}^{\infty} f(t) e^{-j\omega t} dt \approx T_s \sum_{n=0}^{N-1} f(nT_s) e^{-j\omega nT_s}
\]

Avaliando em $\omega = \omega_k = 2\pi k/(NT_s)$:

\[
F(\omega_k) \approx T_s \sum_{n=0}^{N-1} f[n] e^{-j2\pi kn/N} = T_s X[k]
\]

\textbf{Fator $T_s$:} Escala devido ao intervalo de amostragem.

\end{frame}

% ============================================

\subsection{Fast Fourier Transform (FFT)}

\begin{frame}{Complexidade Computacional}

\textbf{Cálculo direto da DFT:}

\[
X[k] = \sum_{n=0}^{N-1} x[n] e^{-j2\pi kn/N}
\]

Para cada $k$, são necessárias:
\begin{itemize}
\item $N$ multiplicações complexas
\item $N-1$ adições complexas
\end{itemize}

Para todos os $N$ valores de $k$:

\textbf{Complexidade: $O(N^2)$}

\vspace{0.3cm}

\textbf{Exemplo:} Para $N = 1024$:

DFT direta: $\approx 1.048.576$ operações

\vspace{0.3cm}

\textbf{Problema:} Proibitivo para $N$ grande (análise em tempo real, processamento de imagens, etc.).

\end{frame}

% ============================================

\begin{frame}{Fast Fourier Transform (FFT)}

A \textbf{FFT} é um algoritmo eficiente para calcular a DFT.

\textbf{Ideia fundamental (Cooley-Tukey):}

Explorar simetrias e periodicidades de $e^{-j2\pi/N}$ (fator de rotação).

Dividir DFT de tamanho $N$ em:
\begin{itemize}
\item Duas DFTs de tamanho $N/2$ (pares e ímpares)
\item Combinar resultados com multiplicações simples
\item Repetir recursivamente
\end{itemize}

\textbf{Complexidade FFT: $O(N\log_2 N)$}

\vspace{0.3cm}

\textbf{Requisito:} $N$ deve ser potência de 2 (para FFT radix-2).

$N = 2^m$ para algum inteiro $m$.

\vspace{0.3cm}

\textbf{Ganho de velocidade:}

Para $N = 1024 = 2^{10}$:

FFT: $\approx 10.240$ operações (100× mais rápido!)

\end{frame}

% ============================================

\begin{frame}{Algoritmo FFT (esboço)}

\textbf{Decomposição butterfly (radix-2):}

\[
X[k] = \sum_{n=0}^{N-1} x[n] W_N^{kn}
\]

onde $W_N = e^{-j2\pi/N}$ (twiddle factor).

Separando índices pares e ímpares:

\[
X[k] = \sum_{r=0}^{N/2-1} x[2r] W_N^{k(2r)} + \sum_{r=0}^{N/2-1} x[2r+1] W_N^{k(2r+1)}
\]

\[
= \sum_{r=0}^{N/2-1} x[2r] W_{N/2}^{kr} + W_N^k \sum_{r=0}^{N/2-1} x[2r+1] W_{N/2}^{kr}
\]

\[
= X_{even}[k] + W_N^k \cdot X_{odd}[k]
\]

Duas DFTs de tamanho $N/2$!

\textbf{Continua recursivamente até DFTs de tamanho 2 (triviais).}

\end{frame}

% ============================================

\begin{frame}[fragile]{Implementação Prática da FFT}

\textbf{Bibliotecas disponíveis:}

\begin{itemize}
\item \textbf{Python:} \texttt{numpy.fft.fft()}, \texttt{scipy.fft}
\item \textbf{MATLAB:} \texttt{fft()}
\item \textbf{C/C++:} FFTW (Fastest Fourier Transform in the West)
\item \textbf{Hardware:} DSPs, FPGAs com blocos FFT dedicados
\end{itemize}

\vspace{0.3cm}

\textbf{Uso típico em Python:}

\begin{verbatim}
import numpy as np

# Sinal no tempo
x = np.array([...])  # N amostras
N = len(x)

# FFT
X = np.fft.fft(x)

# Frequências correspondentes
freqs = np.fft.fftfreq(N, d=Ts)

# Espectro de magnitude
magnitude = np.abs(X)
\end{verbatim}

\end{frame}

% ============================================

\subsection{Fenômenos Práticos}

\begin{frame}{Vazamento Espectral (Spectral Leakage)}

\textbf{Problema:} DFT assume sinal periódico com período $N$.

Se o sinal não é periódico em $N$ ou não contém número inteiro de ciclos:

\textbf{Vazamento espectral:} Energia de uma frequência "vaza" para outras.

\vspace{0.3cm}

\textbf{Causa:}

Truncamento do sinal = multiplicação por janela retangular.

No domínio da frequência: convolução com sinc.

Lóbulos laterais do sinc espalham energia.

\vspace{0.3cm}

\textbf{Consequências:}

\begin{itemize}
\item Picos espectrais alargados
\item Componentes espúrias
\item Resolução reduzida
\end{itemize}

\textbf{Solução:} Janelamento (windowing).

\end{frame}

% ============================================

\begin{frame}{Janelamento (Windowing)}

\textbf{Técnica:} Multiplicar sinal por função janela antes da DFT.

\[
x_w[n] = x[n] \cdot w[n]
\]

\textbf{Janelas comuns:}

\begin{enumerate}
\item \textbf{Retangular:} $w[n] = 1$ (sem janelamento)
   
   Melhor resolução, pior vazamento

\item \textbf{Hanning (Hann):} $w[n] = 0.5[1 - \cos(2\pi n/N)]$
   
   Compromisso, muito usada

\item \textbf{Hamming:} $w[n] = 0.54 - 0.46\cos(2\pi n/N)$
   
   Lóbulos laterais menores

\item \textbf{Blackman:} Combinação de cossenos
   
   Melhor supressão de lóbulos, resolução pior

\item \textbf{Kaiser:} Parametrizável (trade-off ajustável)
\end{enumerate}

\textbf{Compromisso:} Redução de vazamento $\leftrightarrow$ Alargamento do lóbulo principal.

\end{frame}

% ============================================

\begin{frame}{Efeito Cerca (Picket Fence Effect)}

\textbf{Problema:} DFT calcula apenas em $N$ frequências discretas.

Se componente espectral cai entre dois bins de frequência:

\textbf{Efeito:} Amplitude subestimada, distribuída entre bins adjacentes.

\vspace{0.3cm}

\textbf{Analogia:} Ver paisagem através de cerca — perde informação entre ripas.

\vspace{0.3cm}

\textbf{Soluções:}

\begin{enumerate}
\item \textbf{Aumentar $N$:} Mais bins, menor espaçamento $\Delta f$
   
   Requer mais amostras (tempo maior)

\item \textbf{Zero-padding:} Adicionar zeros ao final do sinal
   
   Aumenta $N$ sem coletar mais dados
   
   Interpola espectro (não aumenta informação real!)

\item \textbf{Algoritmos de estimação:} Interpolação parabólica, ajuste de curvas
\end{enumerate}

\end{frame}

% ============================================

\begin{frame}{Zero-Padding}

\textbf{Técnica:} Adicionar zeros ao final da sequência.

Sequência original: $x[0], ..., x[N-1]$

Após zero-padding: $x[0], ..., x[N-1], 0, ..., 0$ (total $M > N$ pontos)

\vspace{0.3cm}

\textbf{Efeito:}

\begin{itemize}
\item DFT de tamanho $M$ (mais bins)
\item Resolução $\Delta f = f_s/M$ (menor que antes)
\item \textbf{Interpola} o espectro entre bins originais
\item \textbf{Não adiciona informação nova!} (apenas interpola)
\end{itemize}

\vspace{0.3cm}

\textbf{Uso:}

\begin{itemize}
\item Visualização mais suave do espectro
\item Facilitarde detecção de picos
\item FFT requer $N = 2^m$: zero-pad para próxima potência de 2
\end{itemize}

\textbf{Nota:} Zero-padding não substitui coleta de mais dados reais!

\end{frame}

% ============================================

\subsection{Exemplos Práticos}

\begin{frame}{Exemplo 1: DFT de Senoide}

\textbf{Sinal:} $x(t) = \cos(2\pi f_0 t)$ com $f_0 = 10$ Hz

\textbf{Amostragem:} $f_s = 100$ Hz, $N = 128$ amostras

\textbf{Parâmetros:}

\begin{itemize}
\item $T_s = 1/f_s = 0.01$ s
\item $T_{obs} = NT_s = 1.28$ s
\item $\Delta f = f_s/N = 100/128 \approx 0.78$ Hz
\end{itemize}

\textbf{Resultado esperado:}

Dois picos em $f = \pm 10$ Hz (ou equivalentemente em $k = 13$ e $k = 115$).

\[
k_0 = \frac{f_0 N}{f_s} = \frac{10 \times 128}{100} = 12.8 \approx 13
\]

\textbf{Observação:} Como $f_0$ não é múltiplo exato de $\Delta f$, haverá algum vazamento espectral.

Uso de janela Hanning reduz vazamento.

\end{frame}

% ============================================

\begin{frame}{Exemplo 2: Análise Espectral de Áudio}

\textbf{Aplicação:} Análise de nota musical (440 Hz = Lá central)

\textbf{Parâmetros:}

\begin{itemize}
\item Taxa de amostragem: $f_s = 44100$ Hz (padrão CD)
\item Tamanho da janela: $N = 4096$ amostras
\item Resolução: $\Delta f = 44100/4096 \approx 10.77$ Hz
\end{itemize}

\textbf{Procedimento:}

\begin{enumerate}
\item Aplicar janela Hamming
\item Calcular FFT (4096 pontos)
\item Encontrar pico dominante
\item Estimar frequência fundamental
\item Identificar harmônicas
\end{enumerate}

\textbf{Resultado:} Pico em $k = 41$ corresponde a:

\[
f = \frac{41 \times 44100}{4096} \approx 441.7 \text{ Hz}
\]

Próximo de 440 Hz (Lá4).

\end{frame}

% ============================================

\begin{frame}{Exemplo 3: Espectrograma}

\textbf{Espectrograma:} Representação tempo-frequência.

\textbf{Ideia:} Dividir sinal longo em janelas curtas, calcular FFT de cada janela.

\textbf{Algoritmo (STFT - Short-Time Fourier Transform):}

\begin{enumerate}
\item Dividir sinal em segmentos sobrepostos
\item Aplicar janela a cada segmento
\item Calcular FFT de cada segmento
\item Empilhar resultados em matriz 2D
\item Visualizar como imagem (cor = magnitude)
\end{enumerate}

\textbf{Parâmetros típicos:}

\begin{itemize}
\item Tamanho da janela: 1024-4096 amostras
\item Sobreposição: 50-75\% (melhor continuidade temporal)
\end{itemize}

\textbf{Aplicações:}

\begin{itemize}
\item Análise de fala (fonemas, formantes)
\item Reconhecimento de padrões sonoros
\item Processamento de áudio (remoção de ruído)
\item Análise de sinais biomédicos (ECG, EEG)
\end{itemize}

\end{frame}

% ============================================

\begin{frame}{Considerações Práticas}

\textbf{Escolha de parâmetros:}

\begin{enumerate}
\item \textbf{Taxa de amostragem $f_s$:}
   
   Pelo menos $2 \times$ frequência máxima de interesse
   
   Margem típica: $f_s = 2.5$ a $4 \times 2f_{max}$

\item \textbf{Número de amostras $N$:}
   
   Determina resolução: $\Delta f = f_s/N$
   
   Para FFT: usar $N = 2^m$ (potência de 2)
   
   Compromisso: resolução vs. localização temporal

\item \textbf{Janela:}
   
   Hanning/Hamming: uso geral
   
   Blackman: melhor supressão de lóbulos
   
   Retangular: apenas se sinal já é periódico em $N$
\end{enumerate}

\vspace{0.3cm}

\textbf{Regra prática:} Para análise espectral, zero-pad para $N \geq 4-8$ vezes o número de amostras reais.

\end{frame}

% ============================================

\begin{frame}{Resumo da Seção 3.9}

\textbf{Conceitos fundamentais:}

\begin{itemize}
\item \textbf{Amostragem:} $f_s \geq 2f_{max}$ (Teorema de Nyquist)
\item \textbf{DFT:} $X[k] = \sum_{n=0}^{N-1} x[n] e^{-j2\pi kn/N}$
\item \textbf{FFT:} Algoritmo rápido, $O(N\log N)$ vs. $O(N^2)$
\end{itemize}

\vspace{0.3cm}

\textbf{Parâmetros importantes:}

\begin{itemize}
\item Resolução: $\Delta f = f_s/N$
\item Faixa: $0$ a $f_s/2$ (Nyquist)
\end{itemize}

\vspace{0.3cm}

\textbf{Fenômenos práticos:}

\begin{itemize}
\item \textbf{Aliasing:} Subamostragem causa distorção
\item \textbf{Vazamento:} Sinal não-periódico espalha energia
\item \textbf{Efeito cerca:} Frequências entre bins subestimadas
\end{itemize}

\vspace{0.3cm}

\textbf{Soluções:}

Janelamento (Hanning, Hamming), zero-padding, sobreamostragem.

\end{frame}
