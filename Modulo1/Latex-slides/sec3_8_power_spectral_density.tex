% ============================================
% SEÇÃO 3.8: POTÊNCIA E DENSIDADE ESPECTRAL DE POTÊNCIA
% ============================================

\subsection{Sinais de Potência}

\begin{frame}{Potência Média de um Sinal}

Para sinais que não têm energia finita (sinais de potência), define-se a \textbf{potência média}:

\[
P = \lim_{T \to \infty} \frac{1}{T} \int_{-T/2}^{T/2} |f(t)|^2 dt
\]

\textbf{Exemplos de sinais de potência:}

\begin{itemize}
\item Sinais periódicos: $\cos(\omega_0 t)$, ondas quadradas
\item Constantes: $f(t) = A$
\item Sinais aleatórios estacionários: ruído branco, processos ergódicos
\end{itemize}

\vspace{0.3cm}

\textbf{Diferença fundamental:}

\begin{center}
\begin{tabular}{|l|c|c|}
\hline
& \textbf{Energia} & \textbf{Potência} \\
\hline
Sinais de energia & Finita & Zero \\
Sinais de potência & Infinita & Finita \\
\hline
\end{tabular}
\end{center}

\textbf{Nota:} $P$ tem unidades de watts (ou normalizadas).

\end{frame}

% ============================================

\begin{frame}{Exemplos: Potência de Sinais}

\textbf{Exemplo 1: Senoide}

$f(t) = A\cos(\omega_0 t)$

\[
P = \lim_{T \to \infty} \frac{1}{T} \int_{-T/2}^{T/2} A^2\cos^2(\omega_0 t) dt
\]

Usando $\cos^2(\theta) = (1 + \cos(2\theta))/2$ e integrando sobre período inteiro:

\[
P = A^2 \cdot \frac{1}{2} = \frac{A^2}{2}
\]

\vspace{0.3cm}

\textbf{Exemplo 2: Constante}

$f(t) = A$

\[
P = \lim_{T \to \infty} \frac{1}{T} \int_{-T/2}^{T/2} A^2 dt = A^2
\]

\vspace{0.3cm}

\textbf{Exemplo 3: Onda quadrada periódica}

Amplitude $\pm A$, período $T_0$: $P = A^2$

\end{frame}

% ============================================

\subsection{Densidade Espectral de Potência}

\begin{frame}{Necessidade da PSD}

\textbf{Problema:} Para sinais de potência, a transformada de Fourier padrão não existe!

Para $f(t) = A\cos(\omega_0 t)$:

\[
\int_{-\infty}^{\infty} |A\cos(\omega_0 t)| dt = \infty
\]

Não satisfaz condição de Dirichlet.

\vspace{0.3cm}

\textbf{Solução:} Usar funções generalizadas (distribuições) ou definir potência no domínio da frequência através da \textbf{Densidade Espectral de Potência (PSD)}.

\textbf{PSD} $S(\omega)$ indica como a potência está distribuída em frequência.

\vspace{0.3cm}

\textbf{Relação fundamental:}

\[
P = \frac{1}{2\pi} \int_{-\infty}^{\infty} S(\omega) d\omega
\]

\end{frame}

% ============================================

\begin{frame}{Definição da PSD}

Para um sinal de potência $f(t)$, define-se:

\textbf{Sinal truncado:}

\[
f_T(t) = \begin{cases}
f(t) & |t| \leq T/2 \\
0 & |t| > T/2
\end{cases}
\]

$f_T(t)$ tem energia finita, logo possui transformada $F_T(\omega)$.

\begin{block}{Densidade Espectral de Potência}
\[
S(\omega) = \lim_{T \to \infty} \frac{1}{T} |F_T(\omega)|^2 = \lim_{T \to \infty} \frac{|F_T(\omega)|^2}{T}
\]
\end{block}

\textbf{Interpretação:}

\begin{itemize}
\item DEE normalizada pelo tempo de observação
\item $S(\omega)$ tem unidades de potência por Hz
\item Indica distribuição de potência no espectro
\end{itemize}

\end{frame}

% ============================================

\begin{frame}{PSD de Sinais Periódicos}

Para sinal periódico com série de Fourier:

\[
f(t) = \sum_{n=-\infty}^{\infty} c_n e^{jn\omega_0 t}
\]

A PSD é:

\begin{block}{PSD de Sinal Periódico}
\[
S(\omega) = 2\pi \sum_{n=-\infty}^{\infty} |c_n|^2 \delta(\omega - n\omega_0)
\]
\end{block}

\textbf{Características:}

\begin{itemize}
\item Espectro discreto (impulsos nas harmônicas)
\item Impulsos localizados em $\omega = n\omega_0$
\item Área de cada impulso: $2\pi|c_n|^2$ (potência naquela harmônica)
\item Potência total: $P = \sum_{n=-\infty}^{\infty} |c_n|^2$
\end{itemize}

\textbf{Significado físico:} Sinal periódico tem potência concentrada em frequências discretas.

\end{frame}

% ============================================

\begin{frame}{Exemplo: PSD de Senoide}

\textbf{Sinal:} $f(t) = A\cos(\omega_0 t)$

Escrevendo em forma exponencial:

\[
f(t) = \frac{A}{2} e^{j\omega_0 t} + \frac{A}{2} e^{-j\omega_0 t}
\]

Coeficientes: $c_1 = c_{-1} = A/2$, $c_n = 0$ para $n \neq \pm 1$

\textbf{PSD:}

\[
S(\omega) = 2\pi \left[\left|\frac{A}{2}\right|^2 \delta(\omega - \omega_0) + \left|\frac{A}{2}\right|^2 \delta(\omega + \omega_0)\right]
\]

\[
= \frac{\pi A^2}{2} [\delta(\omega - \omega_0) + \delta(\omega + \omega_0)]
\]

\textbf{Potência total:}

\[
P = \frac{1}{2\pi} \cdot \frac{\pi A^2}{2} \cdot 2 = \frac{A^2}{2}
\]

Confirma cálculo no tempo!

\end{frame}

% ============================================

\subsection{Autocorrelação e Teorema de Wiener-Khinchin}

\begin{frame}{Função de Autocorrelação para Sinais de Potência}

A \textbf{função de autocorrelação} para um sinal de potência é:

\[
R(\tau) = \lim_{T \to \infty} \frac{1}{T} \int_{-T/2}^{T/2} f(t) f^*(t - \tau) dt
\]

\textbf{Propriedades:}

\begin{itemize}
\item $R(0) = P$ (potência média)
\item $R(\tau) = R^*(-\tau)$ (simetria Hermitiana)
\item Para sinais reais: $R(\tau) = R(-\tau)$ (par)
\item $|R(\tau)| \leq R(0) = P$
\end{itemize}

\vspace{0.3cm}

\textbf{Interpretação física:}

\begin{itemize}
\item Mede correlação temporal do sinal
\item $R(\tau)$ grande $\rightarrow$ valores em $t$ e $t+\tau$ correlacionados
\item $R(\tau)$ pequeno $\rightarrow$ valores descorrelacionados
\end{itemize}

\end{frame}

% ============================================

\begin{frame}{Teorema de Wiener-Khinchin}

\begin{block}{Teorema de Wiener-Khinchin}
A PSD e a função de autocorrelação formam um par de transformadas de Fourier:
\[
R(\tau) \xleftrightarrow{\ft} S(\omega)
\]

Ou seja:
\[
S(\omega) = \int_{-\infty}^{\infty} R(\tau) e^{-j\omega\tau} d\tau
\]
\[
R(\tau) = \frac{1}{2\pi} \int_{-\infty}^{\infty} S(\omega) e^{j\omega\tau} d\omega
\]
\end{block}

\textbf{Importância:}

\begin{itemize}
\item Relaciona propriedades temporais (correlação) com espectrais (potência)
\item PSD pode ser calculada via autocorrelação
\item Fundamental em processamento de sinais aleatórios
\item Base para análise espectral não-paramétrica
\end{itemize}

\end{frame}

% ============================================

\begin{frame}{Demonstração do Teorema (esboço)}

Começando com a definição de PSD:

\[
S(\omega) = \lim_{T \to \infty} \frac{|F_T(\omega)|^2}{T}
\]

Expandindo $|F_T(\omega)|^2 = F_T(\omega) F_T^*(\omega)$:

\[
|F_T(\omega)|^2 = \int_{-T/2}^{T/2} f_T(t) e^{-j\omega t} dt \int_{-T/2}^{T/2} f_T^*(t') e^{j\omega t'} dt'
\]

Combinando as integrais e mudando variável $\tau = t - t'$:

\[
|F_T(\omega)|^2 = \int \int f_T(t) f_T^*(t') e^{-j\omega(t-t')} dt' dt
\]

Após manipulação e tomando limite $T \to \infty$:

\[
S(\omega) = \int_{-\infty}^{\infty} \left[\lim_{T \to \infty} \frac{1}{T} \int f(t) f^*(t-\tau) dt\right] e^{-j\omega\tau} d\tau
\]

\[
= \int_{-\infty}^{\infty} R(\tau) e^{-j\omega\tau} d\tau
\]

\end{frame}

% ============================================

\begin{frame}{Exemplo: Autocorrelação de Senoide}

\textbf{Sinal:} $f(t) = A\cos(\omega_0 t)$

\textbf{Autocorrelação:}

\[
R(\tau) = \lim_{T \to \infty} \frac{1}{T} \int_{-T/2}^{T/2} A\cos(\omega_0 t) \cdot A\cos(\omega_0(t-\tau)) dt
\]

Usando identidade $\cos(a)\cos(b) = \frac{1}{2}[\cos(a-b) + \cos(a+b)]$:

\[
R(\tau) = \lim_{T \to \infty} \frac{A^2}{2T} \int_{-T/2}^{T/2} [\cos(\omega_0 \tau) + \cos(2\omega_0 t - \omega_0\tau)] dt
\]

O segundo termo integra a zero sobre período completo:

\[
R(\tau) = \frac{A^2}{2} \cos(\omega_0 \tau)
\]

\textbf{Transformada:}

\[
S(\omega) = \frac{A^2}{2} \FT{\cos(\omega_0 \tau)} = \frac{\pi A^2}{2}[\delta(\omega - \omega_0) + \delta(\omega + \omega_0)]
\]

Confirma resultado anterior!

\end{frame}

% ============================================

\subsection{PSD na Saída de Sistemas LTI}

\begin{frame}{Transformação de PSD por Sistema LTI}

\textbf{Sistema:} Entrada $x(t)$ com PSD $S_x(\omega)$, saída $y(t)$ com PSD $S_y(\omega)$.

\textbf{Relação na frequência:} $Y(\omega) = H(\omega) X(\omega)$

\textbf{Derivação:}

Para sinal truncado: $Y_T(\omega) = H(\omega) X_T(\omega)$

\[
|Y_T(\omega)|^2 = |H(\omega)|^2 |X_T(\omega)|^2
\]

Tomando limite:

\[
S_y(\omega) = \lim_{T \to \infty} \frac{|Y_T(\omega)|^2}{T} = |H(\omega)|^2 \lim_{T \to \infty} \frac{|X_T(\omega)|^2}{T}
\]

\begin{block}{Relação Entrada-Saída para PSD}
\[
\boxed{S_y(\omega) = |H(\omega)|^2 S_x(\omega)}
\]
\end{block}

\textbf{Idêntico à relação para DEE!}

\end{frame}

% ============================================

\begin{frame}{Potência na Saída de Sistema LTI}

A potência na saída é:

\[
P_y = \frac{1}{2\pi} \int_{-\infty}^{\infty} S_y(\omega) d\omega
\]

Substituindo $S_y(\omega) = |H(\omega)|^2 S_x(\omega)$:

\[
P_y = \frac{1}{2\pi} \int_{-\infty}^{\infty} |H(\omega)|^2 S_x(\omega) d\omega
\]

\textbf{Interpretação:}

\begin{itemize}
\item Cada componente espectral da entrada é filtrada por $|H(\omega)|^2$
\item Frequências amplificadas pelo sistema contribuem mais à potência de saída
\item Frequências atenuadas contribuem menos
\end{itemize}

\vspace{0.3cm}

\textbf{Aplicação importante:} Análise de ruído em sistemas de comunicação.

\end{frame}

% ============================================

\subsection{Ruído Branco}

\begin{frame}{Ruído Branco}

\textbf{Ruído branco ideal:} Processo aleatório com PSD constante para todas as frequências.

\begin{block}{PSD de Ruído Branco}
\[
S_n(\omega) = \frac{N_0}{2} \quad \text{para todo} \quad \omega
\]
\end{block}

onde $N_0$ é a densidade espectral de potência bilateral.

\vspace{0.3cm}

\textbf{Características:}

\begin{itemize}
\item Potência infinita: $P = \int_{-\infty}^{\infty} (N_0/2)/(2\pi) d\omega = \infty$
\item Todas as frequências igualmente representadas
\item Analogia: Luz branca contém todas as cores
\item Modelo idealizado (fisicamente irrealizável)
\end{itemize}

\textbf{Autocorrelação:}

\[
R_n(\tau) = \frac{N_0}{2} \delta(\tau)
\]

Amostras em tempos diferentes são completamente descorrelacionadas!

\end{frame}

% ============================================

\begin{frame}{Ruído Branco Filtrado}

\textbf{Situação prática:} Ruído branco passa por filtro (canal, receptor).

\textbf{Entrada:} Ruído branco $n(t)$ com $S_n(\omega) = N_0/2$

\textbf{Saída:} $y(t) = n(t) \conv h(t)$

\textbf{PSD da saída:}

\[
S_y(\omega) = |H(\omega)|^2 \cdot \frac{N_0}{2}
\]

\textbf{Potência do ruído na saída:}

\[
P_y = \frac{1}{2\pi} \int_{-\infty}^{\infty} |H(\omega)|^2 \frac{N_0}{2} d\omega = \frac{N_0}{4\pi} \int_{-\infty}^{\infty} |H(\omega)|^2 d\omega
\]

Pela relação de Parseval:

\[
P_y = \frac{N_0}{2} \int_{-\infty}^{\infty} |h(t)|^2 dt
\]

\textbf{Ruído filtrado já não é branco!} PSD tem forma de $|H(\omega)|^2$.

\end{frame}

% ============================================

\begin{frame}{Exemplo: Ruído Branco em Filtro RC}

\textbf{Filtro:} RC passa-baixas com $H(\omega) = 1/(1 + j\omega RC)$

\textbf{Entrada:} Ruído branco $S_n(\omega) = N_0/2$

\textbf{PSD na saída:}

\[
S_y(\omega) = \frac{N_0/2}{1 + (\omega RC)^2}
\]

\textbf{Potência do ruído na saída:}

\[
P_y = \frac{1}{2\pi} \int_{-\infty}^{\infty} \frac{N_0/2}{1 + (\omega RC)^2} d\omega
\]

Com $\omega_c = 1/(RC)$:

\[
P_y = \frac{N_0}{4\pi} \cdot \frac{2\pi}{\omega_c} = \frac{N_0}{4\omega_c} = \frac{N_0}{4} \cdot RC = \frac{N_0 \cdot BW}{2}
\]

onde $BW = 1/(4RC)$ é a largura de banda de ruído do filtro.

\textbf{Conclusão:} Potência de ruído proporcional à largura de banda!

\end{frame}

% ============================================

\subsection{Relação Sinal-Ruído (SNR)}

\begin{frame}{Relação Sinal-Ruído}

A \textbf{Relação Sinal-Ruído (SNR)} quantifica qualidade do sinal:

\[
\text{SNR} = \frac{P_{\text{sinal}}}{P_{\text{ruído}}}
\]

Em decibéis:

\[
\text{SNR}_{dB} = 10\log_{10}\left(\frac{P_{\text{sinal}}}{P_{\text{ruído}}}\right)
\]

\vspace{0.3cm}

\textbf{Importância:}

\begin{itemize}
\item Medida fundamental de qualidade em comunicações
\item Determina taxa de erro e capacidade do canal
\item SNR alta: comunicação confiável
\item SNR baixa: erros frequentes, degradação
\end{itemize}

\vspace{0.3cm}

\textbf{Valores típicos:}

\begin{itemize}
\item Telefonia analógica: SNR $>$ 30 dB
\item Comunicação digital: SNR $>$ 10-15 dB (depende da modulação)
\item Rádio FM: SNR $>$ 40 dB para boa qualidade
\end{itemize}

\end{frame}

% ============================================

\begin{frame}{SNR em Sistemas de Banda Limitada}

\textbf{Sistema:} Sinal $s(t)$ com potência $P_s$ + ruído branco $n(t)$ com PSD $N_0/2$.

\textbf{Após filtro passa-baixas ideal de largura $W$:}

Potência do sinal: $P_s$ (assumindo sinal banda-limitada em $W$)

Potência do ruído:

\[
P_n = \frac{1}{2\pi} \int_{-W}^{W} \frac{N_0}{2} d\omega = \frac{N_0 \cdot 2W}{4\pi} = \frac{N_0 W}{2\pi}
\]

Em Hz ($f = \omega/2\pi$):

\[
P_n = N_0 B
\]

onde $B = W/(2\pi)$ é a largura de banda em Hz.

\textbf{SNR:}

\[
\text{SNR} = \frac{P_s}{N_0 B}
\]

\textbf{Compromisso:} Aumentar banda $B$ aumenta capacidade mas também aumenta ruído!

\end{frame}

% ============================================

\subsection{Exemplo Integrado}

\begin{frame}{Exemplo: Comunicação em Canal Ruidoso}

\textbf{Sistema:}

\begin{itemize}
\item Sinal transmitido: $s(t) = A\cos(\omega_c t)$, potência $P_s = A^2/2$
\item Canal: Atenuação $\alpha$, ruído branco aditivo $N_0/2$
\item Filtro receptor: Passa-faixas centrado em $\omega_c$, largura $2B$
\end{itemize}

\textbf{Sinal recebido (antes do filtro):}

\[
r(t) = \alpha s(t) + n(t)
\]

\textbf{Após filtro:}

Potência do sinal: $P_{s,out} = \alpha^2 P_s = \alpha^2 A^2/2$

Potência do ruído: $P_{n,out} = N_0 B$ (banda bilateral $2B$, mas simetria)

\textbf{SNR na saída:}

\[
\text{SNR}_{out} = \frac{\alpha^2 A^2/2}{N_0 B} = \frac{\alpha^2 A^2}{2N_0 B}
\]

\end{frame}

% ============================================

\begin{frame}{Resumo da Seção 3.8}

\textbf{Conceitos de potência:}

\begin{itemize}
\item \textbf{Potência média:} $P = \lim_{T\to\infty} \frac{1}{T}\int |f(t)|^2 dt$
\item Para sinais periódicos: $P = \sum |c_n|^2$
\end{itemize}

\vspace{0.3cm}

\textbf{Densidade Espectral de Potência (PSD):}

\begin{itemize}
\item $S(\omega)$: distribuição de potência em frequência
\item $P = \frac{1}{2\pi}\int S(\omega) d\omega$
\item Sinais periódicos: espectro discreto (impulsos)
\end{itemize}

\vspace{0.3cm}

\textbf{Teorema de Wiener-Khinchin:} $R(\tau) \xleftrightarrow{\ft} S(\omega)$

\vspace{0.3cm}

\textbf{Sistemas LTI:} $S_y(\omega) = |H(\omega)|^2 S_x(\omega)$

\vspace{0.3cm}

\textbf{Ruído branco:}

\begin{itemize}
\item PSD constante: $S_n = N_0/2$
\item Após filtragem: potência $\propto$ largura de banda
\item SNR = $P_s/(N_0 B)$
\end{itemize}

\end{frame}
