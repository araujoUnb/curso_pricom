% ============================================
% SEÇÃO 3.3: PROPRIEDADES DA TRANSFORMADA DE FOURIER
% ============================================

\subsection{Introdução}

\begin{frame}{Propriedades da Transformada de Fourier}

As propriedades da Transformada de Fourier permitem:

\begin{itemize}
\item Derivar novas transformadas a partir de conhecidas
\item Simplificar cálculos complexos
\item Entender relações fundamentais entre tempo e frequência
\item Analisar sistemas e operações sobre sinais
\end{itemize}

\vspace{0.5cm}

\textbf{Propriedades principais a serem estudadas:}

\begin{enumerate}
\item Linearidade
\item Deslocamento no tempo e na frequência
\item Escala temporal
\item Convolução (tempo e frequência)
\item Derivação e integração
\item Dualidade e simetrias
\item Teorema de Parseval
\end{enumerate}

\end{frame}

% ============================================

\subsection{Linearidade}

\begin{frame}{Propriedade de Linearidade}

\begin{block}{Propriedade 1: Linearidade}
Se $f_1(t) \xleftrightarrow{\ft} F_1(\omega)$ e $f_2(t) \xleftrightarrow{\ft} F_2(\omega)$, então:
\[
af_1(t) + bf_2(t) \xleftrightarrow{\ft} aF_1(\omega) + bF_2(\omega)
\]
para quaisquer constantes $a$ e $b$.
\end{block}

\textbf{Demonstração:}

\[
\FT{af_1(t) + bf_2(t)} = \int_{-\infty}^{\infty} [af_1(t) + bf_2(t)] e^{-j\omega t} dt
\]

\[
= a\int_{-\infty}^{\infty} f_1(t) e^{-j\omega t} dt + b\int_{-\infty}^{\infty} f_2(t) e^{-j\omega t} dt
\]

\[
= aF_1(\omega) + bF_2(\omega)
\]

\textbf{Conclusão:} A transformada de Fourier é um \textbf{operador linear}.

\end{frame}

% ============================================

\begin{frame}{Exemplo: Linearidade}

\textbf{Problema:} Encontrar a transformada de $f(t) = 3e^{-2t}u(t) + 5e^{-4|t|}$

\textbf{Solução:}

Sabemos que:
\begin{itemize}
\item $e^{-2t}u(t) \xleftrightarrow{\ft} \frac{1}{2+j\omega}$
\item $e^{-4|t|} \xleftrightarrow{\ft} \frac{8}{16+\omega^2}$
\end{itemize}

Pela linearidade:

\[
F(\omega) = 3 \cdot \frac{1}{2+j\omega} + 5 \cdot \frac{8}{16+\omega^2}
\]

\[
= \frac{3}{2+j\omega} + \frac{40}{16+\omega^2}
\]

\textbf{Sem linearidade:} Teríamos que calcular integrais complexas!

\end{frame}

% ============================================

\subsection{Deslocamento Temporal}

\begin{frame}{Propriedade de Deslocamento no Tempo}

\begin{block}{Propriedade 2: Deslocamento Temporal}
\[
f(t - t_0) \xleftrightarrow{\ft} F(\omega)\, e^{-j\omega t_0}
\]
\end{block}

\textbf{Demonstração:} Mudança de variável $\tau = t - t_0$:

\[
\FT{f(t - t_0)} = \int_{-\infty}^{\infty} f(\tau)\, e^{-j\omega(\tau + t_0)}\, d\tau
= e^{-j\omega t_0}\!\int_{-\infty}^{\infty} f(\tau)\, e^{-j\omega\tau}\, d\tau
= F(\omega)\,e^{-j\omega t_0}
\]

\textbf{Observações:}
\begin{itemize}
\item Magnitude não muda: $|F(\omega)\,e^{-j\omega t_0}| = |F(\omega)|$
\item Fase adicionada é \textbf{linear}: $\Delta\phi(\omega) = -\omega t_0$
\item Atraso no tempo $\leftrightarrow$ fase linear na frequência
\end{itemize}

\textbf{Exemplo:} $\rect\!\left(\dfrac{t-t_0}{\tau}\right) \xleftrightarrow{\ft} \tau\sinc\!\left(\dfrac{\omega\tau}{2\pi}\right) e^{-j\omega t_0}$

\end{frame}

% ============================================

\begin{frame}{Interpretação do Deslocamento Temporal}

\[
f(t - t_0) \xleftrightarrow{\ft} F(\omega) e^{-j\omega t_0}
\]

\textbf{Observações importantes:}

\begin{itemize}
\item \textbf{Magnitude não muda:} $|F(\omega) e^{-j\omega t_0}| = |F(\omega)|$
\item \textbf{Fase é alterada:} $\angle[F(\omega) e^{-j\omega t_0}] = \angle F(\omega) - \omega t_0$
\item Deslocamento no tempo $\rightarrow$ mudança de fase linear na frequência
\item A inclinação da fase adicional é $-t_0$
\end{itemize}

\vspace{0.3cm}

\textbf{Exemplo prático:} $\rect\left(\frac{t-t_0}{\tau}\right) \xleftrightarrow{\ft} \tau \sinc\left(\frac{\omega\tau}{2\pi}\right) e^{-j\omega t_0}$

Um pulso atrasado tem o mesmo espectro de magnitude, mas fase alterada.

\end{frame}

% ============================================

\subsection{Deslocamento em Frequência}

\begin{frame}{Propriedade de Deslocamento em Frequência (Modulação)}

\begin{block}{Propriedade 3: Deslocamento em Frequência}
Se $f(t) \xleftrightarrow{\ft} F(\omega)$, então:
\[
f(t) e^{j\omega_0 t} \xleftrightarrow{\ft} F(\omega - \omega_0)
\]
\end{block}

\textbf{Demonstração:}

\[
\FT{f(t) e^{j\omega_0 t}} = \int_{-\infty}^{\infty} f(t) e^{j\omega_0 t} e^{-j\omega t} dt
\]

\[
= \int_{-\infty}^{\infty} f(t) e^{-j(\omega - \omega_0) t} dt = F(\omega - \omega_0)
\]

\textbf{Interpretação:} Multiplicar por $e^{j\omega_0 t}$ no tempo desloca o espectro em $\omega_0$.

\textbf{Aplicação fundamental:} \textbf{Modulação} em sistemas de comunicação!

\end{frame}

% ============================================

\begin{frame}{Modulação AM e o Teorema da Modulação}

Usando $\cos(\omega_0 t) = \dfrac{e^{j\omega_0 t} + e^{-j\omega_0 t}}{2}$ e a linearidade:

\[
\FT{f(t) \cos(\omega_0 t)}
= \tfrac{1}{2}\FT{f(t)e^{j\omega_0 t}} + \tfrac{1}{2}\FT{f(t)e^{-j\omega_0 t}}
\]

\begin{block}{Teorema da Modulação}
\[
f(t) \cos(\omega_0 t) \xleftrightarrow{\ft} \frac{1}{2}\bigl[F(\omega - \omega_0) + F(\omega + \omega_0)\bigr]
\]
\end{block}

\textbf{Interpretação:}
\begin{itemize}
\item Multiplicar por cosseno cria \textbf{duas cópias} do espectro
\item Cada cópia deslocada para $+\omega_0$ e $-\omega_0$
\item Base da \textbf{modulação AM}: portadora na frequência $\omega_0$
\end{itemize}

\end{frame}

% ============================================

\subsection{Escala Temporal}

\begin{frame}{Propriedade de Escala Temporal}

\begin{block}{Propriedade 4: Escala Temporal}
Se $f(t) \xleftrightarrow{\ft} F(\omega)$ e $a \neq 0$, então:
\[
f(at) \xleftrightarrow{\ft} \frac{1}{|a|} F\!\left(\frac{\omega}{a}\right)
\]
\end{block}

\textbf{Demonstração} ($a>0$): substituindo $\tau = at$, $dt = d\tau/a$:

\begin{align*}
\FT{f(at)}
  &= \int_{-\infty}^{\infty} f(at)\,e^{-j\omega t}\,dt
   = \frac{1}{a}\int_{-\infty}^{\infty} f(\tau)\,e^{-j(\omega/a)\tau}\,d\tau \\[4pt]
  &= \frac{1}{a}\,F\!\left(\frac{\omega}{a}\right)
\end{align*}

Para $a < 0$ os limites invertem, dando o fator $1/|a|$.

\end{frame}

% ============================================

\begin{frame}{Interpretação da Escala Temporal}

\[
f(at) \xleftrightarrow{\ft} \frac{1}{|a|} F\left(\frac{\omega}{a}\right)
\]

\textbf{Casos importantes:}

\begin{enumerate}
\item \textbf{Compressão no tempo} ($a > 1$):
   \begin{itemize}
   \item Sinal mais rápido/estreito no tempo
   \item Espectro mais largo/expandido em frequência
   \item Amplitude espectral reduzida por $1/a$
   \end{itemize}

\item \textbf{Expansão no tempo} ($0 < a < 1$):
   \begin{itemize}
   \item Sinal mais lento/largo no tempo
   \item Espectro mais estreito/comprimido em frequência
   \item Amplitude espectral aumentada por $1/a$
   \end{itemize}

\item \textbf{Reversão temporal} ($a = -1$):
   \[
   f(-t) \xleftrightarrow{\ft} F(-\omega)
   \]
\end{enumerate}

\textbf{Princípio fundamental:} Compromisso tempo-frequência!

\end{frame}

% ============================================

\begin{frame}{Exemplo: Escala Temporal}

\textbf{Problema:} Se $\rect(t) \xleftrightarrow{\ft} \sinc(\omega/2\pi)$, encontre a transformada de $\rect(t/5)$.

\textbf{Solução:}

Temos $f(t) = \rect(t)$ e queremos $\FT{\rect(t/5)} = \FT{f(t/5)}$.

Isso corresponde a $a = 1/5$ na propriedade de escala:

\[
f(t/5) \xleftrightarrow{\ft} \frac{1}{|1/5|} F\left(\frac{\omega}{1/5}\right) = 5 F(5\omega)
\]

Como $F(\omega) = \sinc(\omega/2\pi)$:

\[
\FT{\rect(t/5)} = 5 \sinc(5\omega/2\pi) = 5 \sinc(\omega/\frac{2\pi}{5})
\]

\textbf{Verificação:} Pulso 5× mais largo $\rightarrow$ espectro 5× mais estreito e 5× mais alto.

\end{frame}

% ============================================

\subsection{Convolução}

\begin{frame}{Convolução no Tempo}

A \textbf{convolução} de dois sinais:
\[
[f_1 \conv f_2](t) = \int_{-\infty}^{\infty} f_1(\tau) f_2(t - \tau) d\tau
\]

\begin{block}{Propriedade 5: Convolução no Tempo}
Se $f_1(t) \xleftrightarrow{\ft} F_1(\omega)$ e $f_2(t) \xleftrightarrow{\ft} F_2(\omega)$, então:
\[
f_1(t) \conv f_2(t) \xleftrightarrow{\ft} F_1(\omega) \cdot F_2(\omega)
\]
\end{block}

\textbf{Significado:} Convolução no tempo $\leftrightarrow$ multiplicação na frequência.

\textbf{Importância:} A resposta de um sistema LTI é a convolução da entrada com a resposta ao impulso:
\[
y(t) = x(t) \conv h(t) \quad \Rightarrow \quad Y(\omega) = X(\omega) H(\omega)
\]

\end{frame}

% ============================================

\begin{frame}{Demonstração: Convolução no Tempo}

Queremos mostrar que $\FT{f_1(t) \conv f_2(t)} = F_1(\omega) \cdot F_2(\omega)$.

\[
\FT{f_1 \conv f_2} = \int_{-\infty}^{\infty} \left[\int_{-\infty}^{\infty} f_1(\tau) f_2(t - \tau) d\tau\right] e^{-j\omega t} dt
\]

Trocando a ordem de integração:

\[
= \int_{-\infty}^{\infty} f_1(\tau) \left[\int_{-\infty}^{\infty} f_2(t - \tau) e^{-j\omega t} dt\right] d\tau
\]

A integral interna é a transformada de $f_2(t - \tau)$. Pela propriedade de deslocamento:

\[
\int_{-\infty}^{\infty} f_2(t - \tau) e^{-j\omega t} dt = F_2(\omega) e^{-j\omega\tau}
\]

Portanto:

\[
= \int_{-\infty}^{\infty} f_1(\tau) F_2(\omega) e^{-j\omega\tau} d\tau = F_2(\omega) \int_{-\infty}^{\infty} f_1(\tau) e^{-j\omega\tau} d\tau = F_2(\omega) F_1(\omega)
\]

\end{frame}

% ============================================

\begin{frame}{Visualização: Convolução no Tempo = Multiplicação em Frequência}

\begin{center}
\includegraphics[width=\figFull, height=0.72\textheight, keepaspectratio]{figures/cap3/convolution_example}
\end{center}

\vspace{-0.2cm}
\begin{block}{Resultado}
Calcular via convolução no tempo \textbf{ou} multiplicação na frequência dá o mesmo resultado.
\end{block}

\end{frame}

% ============================================

\begin{frame}{Convolução em Frequência (Multiplicação no Tempo)}

\begin{block}{Propriedade 6: Convolução em Frequência}
Se $f_1(t) \xleftrightarrow{\ft} F_1(\omega)$ e $f_2(t) \xleftrightarrow{\ft} F_2(\omega)$, então:
\[
f_1(t) \cdot f_2(t) \xleftrightarrow{\ft} \frac{1}{2\pi} [F_1(\omega) \conv F_2(\omega)]
\]
\end{block}

onde a convolução em frequência é:
\[
[F_1 \conv F_2](\omega) = \int_{-\infty}^{\infty} F_1(\lambda) F_2(\omega - \lambda) d\lambda
\]

\textbf{Demonstração:} Similar à anterior, usando a transformada inversa.

\textbf{Dualidade com a propriedade anterior:}
\begin{itemize}
\item Tempo: convolução $\leftrightarrow$ Frequência: multiplicação
\item Tempo: multiplicação $\leftrightarrow$ Frequência: convolução (com fator $1/2\pi$)
\end{itemize}

\end{frame}

% ============================================

\subsection{Derivação e Integração}

\begin{frame}{Propriedade de Derivação}

\begin{block}{Propriedade 7: Derivação no Tempo}
Se $f(t) \xleftrightarrow{\ft} F(\omega)$, então:
\[
\frac{df(t)}{dt} \xleftrightarrow{\ft} j\omega F(\omega)
\]
\end{block}

\textbf{Demonstração:}

Começamos com:
\[
f(t) = \frac{1}{2\pi} \int_{-\infty}^{\infty} F(\omega) e^{j\omega t} d\omega
\]

Derivando ambos os lados em relação a $t$:

\[
\frac{df(t)}{dt} = \frac{1}{2\pi} \int_{-\infty}^{\infty} F(\omega) \frac{d}{dt}[e^{j\omega t}] d\omega = \frac{1}{2\pi} \int_{-\infty}^{\infty} F(\omega) \cdot j\omega e^{j\omega t} d\omega
\]

\[
= \frac{1}{2\pi} \int_{-\infty}^{\infty} [j\omega F(\omega)] e^{j\omega t} d\omega
\]

Portanto: $\FT{df/dt} = j\omega F(\omega)$

\end{frame}

% ============================================

\begin{frame}{Interpretação e Generalização da Derivação}

\[
\frac{df(t)}{dt} \xleftrightarrow{\ft} j\omega F(\omega)
\]

\textbf{Interpretação:}
\begin{itemize}
\item Derivar no tempo = multiplicar por $j\omega$ na frequência
\item Componentes de alta frequência são enfatizadas (proporcional a $\omega$)
\item Componente DC é eliminada ($\omega = 0$)
\end{itemize}

\textbf{Generalização para n-ésima derivada:}
\[
\frac{d^n f(t)}{dt^n} \xleftrightarrow{\ft} (j\omega)^n F(\omega)
\]

\textbf{Aplicação:} Resolver equações diferenciais!

Exemplo: $\frac{dy}{dt} + ay = x(t)$ torna-se $(j\omega + a)Y(\omega) = X(\omega)$

Solução: $Y(\omega) = \frac{X(\omega)}{j\omega + a}$

\end{frame}

% ============================================

\begin{frame}{Propriedade de Integração}

\begin{block}{Propriedade 8: Integração no Tempo}
Se $f(t) \xleftrightarrow{\ft} F(\omega)$ e $F(0) = 0$, então:
\[
\int_{-\infty}^{t} f(\tau) d\tau \xleftrightarrow{\ft} \frac{F(\omega)}{j\omega}
\]

Se $F(0) \neq 0$:
\[
\int_{-\infty}^{t} f(\tau) d\tau \xleftrightarrow{\ft} \frac{F(\omega)}{j\omega} + \pi F(0) \delta(\omega)
\]
\end{block}

\textbf{Interpretação:}
\begin{itemize}
\item Integrar no tempo = dividir por $j\omega$ na frequência
\item Componentes de baixa frequência são enfatizadas (proporcional a $1/\omega$)
\item Pode adicionar componente DC se a área sob $f(t)$ for não-zero
\end{itemize}

\textbf{Relação com derivação:} Operações inversas no tempo $\leftrightarrow$ operações inversas na frequência.

\end{frame}

% ============================================

\begin{frame}{Exemplo: Derivação e Integração}

\textbf{Problema:} Sabendo que $e^{-at}u(t) \xleftrightarrow{\ft} \frac{1}{a+j\omega}$, encontre a transformada de $te^{-at}u(t)$.

\textbf{Solução:} Use a propriedade de derivação na frequência (dual da derivação no tempo):

\[
(-jt) f(t) \xleftrightarrow{\ft} \frac{dF(\omega)}{d\omega}
\]

Portanto:
\[
t f(t) \xleftrightarrow{\ft} j \frac{dF(\omega)}{d\omega}
\]

Com $f(t) = e^{-at}u(t)$ e $F(\omega) = \frac{1}{a+j\omega}$:

\[
\frac{dF(\omega)}{d\omega} = \frac{d}{d\omega}\left[\frac{1}{a+j\omega}\right] = \frac{-j}{(a+j\omega)^2}
\]

\[
\FT{te^{-at}u(t)} = j \cdot \frac{-j}{(a+j\omega)^2} = \frac{1}{(a+j\omega)^2}
\]

\end{frame}

% ============================================

\subsection{Dualidade}

\begin{frame}{Propriedade de Dualidade}

\begin{block}{Propriedade 9: Dualidade}
Se $f(t) \xleftrightarrow{\ft} F(\omega)$, então:
\[
F(t) \xleftrightarrow{\ft} 2\pi f(-\omega)
\]
\end{block}

\textbf{Demonstração:}

Começamos com a transformada inversa:
\[
f(t) = \frac{1}{2\pi} \int_{-\infty}^{\infty} F(\omega) e^{j\omega t} d\omega
\]

Trocando $t$ e $\omega$:
\[
f(\omega) = \frac{1}{2\pi} \int_{-\infty}^{\infty} F(t) e^{j\omega t} dt
\]

Multiplicando por $2\pi$ e substituindo $\omega \to -\omega$:
\[
2\pi f(-\omega) = \int_{-\infty}^{\infty} F(t) e^{-j\omega t} dt = \FT{F(t)}
\]

\end{frame}

% ============================================

\begin{frame}{Exemplos de Dualidade}

A dualidade permite obter novos pares de transformadas:

\textbf{Exemplo 1:}
\[
\delta(t) \xleftrightarrow{\ft} 1 \quad \Rightarrow \quad 1 \xleftrightarrow{\ft} 2\pi\delta(-\omega) = 2\pi\delta(\omega)
\]

\textbf{Exemplo 2:}
\[
\rect(t/\tau) \xleftrightarrow{\ft} \tau\sinc(\omega\tau/2\pi)
\]

Pela dualidade:
\[
\tau\sinc(t\tau/2\pi) \xleftrightarrow{\ft} 2\pi\rect(-\omega/\tau) = 2\pi\rect(\omega/\tau)
\]

Simplificando com $\tau = 2\pi/W$:
\[
\frac{2W}{\pi}\sinc(Wt/\pi) \xleftrightarrow{\ft} 2\pi\rect(\omega\pi/2W)
\]

\textbf{Conclusão:} Função sinc no tempo $\leftrightarrow$ retângulo na frequência (banda limitada!)

\end{frame}

% ============================================

\subsection{Simetrias}

\begin{frame}{Propriedades de Simetria para Sinais Reais}

Para um sinal \textbf{real} $f(t) = f^*(t)$, a transformada possui simetrias importantes:

\begin{block}{Simetria Hermitiana}
Se $f(t)$ é real, então:
\[
F(-\omega) = F^*(\omega)
\]
\end{block}

\textbf{Consequências:}

\begin{itemize}
\item \textbf{Parte real:} $\Real\{F(-\omega)\} = \Real\{F(\omega)\}$ (par)
\item \textbf{Parte imaginária:} $\Imag\{F(-\omega)\} = -\Imag\{F(\omega)\}$ (ímpar)
\item \textbf{Magnitude:} $|F(-\omega)| = |F(\omega)|$ (par)
\item \textbf{Fase:} $\angle F(-\omega) = -\angle F(\omega)$ (ímpar)
\end{itemize}

\vspace{0.3cm}

\textbf{Implicação prática:} Para sinais reais, basta conhecer $F(\omega)$ para $\omega \geq 0$!

\end{frame}

% ============================================

\begin{frame}{Simetrias para Sinais Pares e Ímpares}

\textbf{Se} $f(t)$ \textbf{é real e par} ($f(-t) = f(t)$):

\[
F(\omega) = \int_{-\infty}^{\infty} f(t) e^{-j\omega t} dt = \int_{-\infty}^{\infty} f(t) \cos(\omega t) dt
\]

A componente imaginária se cancela, então $F(\omega)$ é \textbf{real e par}.

\vspace{0.3cm}

\textbf{Se} $f(t)$ \textbf{é real e ímpar} ($f(-t) = -f(t)$):

\[
F(\omega) = -j \int_{-\infty}^{\infty} f(t) \sin(\omega t) dt
\]

A componente real se cancela, então $F(\omega)$ é \textbf{imaginário puro e ímpar}.

\vspace{0.3cm}

\textbf{Tabela resumo:}

\begin{center}
\begin{tabular}{|c|c|}
\hline
$f(t)$ & $F(\omega)$ \\
\hline
Real e par & Real e par \\
Real e ímpar & Imaginário puro e ímpar \\
\hline
\end{tabular}
\end{center}

\end{frame}

% ============================================

\subsection{Teorema de Parseval}

\begin{frame}{Teorema de Parseval (Conservação de Energia)}

\begin{block}{Teorema de Parseval}
Se $f(t) \xleftrightarrow{\ft} F(\omega)$, então:
\[
\int_{-\infty}^{\infty} |f(t)|^2 dt = \frac{1}{2\pi} \int_{-\infty}^{\infty} |F(\omega)|^2 d\omega
\]
\end{block}

\textbf{Interpretação:}
\begin{itemize}
\item Lado esquerdo: \textbf{energia total no domínio do tempo}
\item Lado direito: \textbf{energia total no domínio da frequência}
\item A energia é conservada entre os dois domínios!
\end{itemize}

\vspace{0.3cm}

\textbf{Generalização (Teorema de Rayleigh):}
\[
\int_{-\infty}^{\infty} f_1(t) f_2^*(t) dt = \frac{1}{2\pi} \int_{-\infty}^{\infty} F_1(\omega) F_2^*(\omega) d\omega
\]

\end{frame}

% ============================================

\begin{frame}{Demonstração do Teorema de Parseval}

Começamos com:
\[
\int_{-\infty}^{\infty} |f(t)|^2 dt = \int_{-\infty}^{\infty} f(t) f^*(t) dt
\]

Substituindo a transformada inversa para $f^*(t)$:
\[
f^*(t) = \left[\frac{1}{2\pi} \int_{-\infty}^{\infty} F(\omega) e^{j\omega t} d\omega\right]^* = \frac{1}{2\pi} \int_{-\infty}^{\infty} F^*(\omega) e^{-j\omega t} d\omega
\]

Então:
\[
\int_{-\infty}^{\infty} |f(t)|^2 dt = \int_{-\infty}^{\infty} f(t) \left[\frac{1}{2\pi} \int_{-\infty}^{\infty} F^*(\omega) e^{-j\omega t} d\omega\right] dt
\]

Trocando a ordem:
\[
= \frac{1}{2\pi} \int_{-\infty}^{\infty} F^*(\omega) \left[\int_{-\infty}^{\infty} f(t) e^{-j\omega t} dt\right] d\omega = \frac{1}{2\pi} \int_{-\infty}^{\infty} F^*(\omega) F(\omega) d\omega
\]

\[
= \frac{1}{2\pi} \int_{-\infty}^{\infty} |F(\omega)|^2 d\omega
\]

\end{frame}

% ============================================

\begin{frame}{Tabela Resumo: Propriedades (I)}

\begin{table}
\small
\begin{tabular}{|l|c|}
\hline
\textbf{Propriedade} & \textbf{Relação} \\
\hline
Linearidade & $af_1(t) + bf_2(t) \xleftrightarrow{\ft} aF_1(\omega) + bF_2(\omega)$ \\
\hline
Deslocamento temporal & $f(t - t_0) \xleftrightarrow{\ft} F(\omega)\,e^{-j\omega t_0}$ \\
\hline
Deslocamento em freq. & $f(t)\,e^{j\omega_0 t} \xleftrightarrow{\ft} F(\omega - \omega_0)$ \\
\hline
Escala temporal & $f(at) \xleftrightarrow{\ft} \dfrac{1}{|a|}F\!\left(\dfrac{\omega}{a}\right)$ \\[4pt]
\hline
Convolução no tempo & $f_1(t) \conv f_2(t) \xleftrightarrow{\ft} F_1(\omega)\cdot F_2(\omega)$ \\
\hline
Multiplicação no tempo & $f_1(t)\cdot f_2(t) \xleftrightarrow{\ft} \dfrac{1}{2\pi}[F_1 \conv F_2](\omega)$ \\[4pt]
\hline
\end{tabular}
\end{table}

\end{frame}

% ============================================

\begin{frame}{Tabela Resumo: Propriedades (II)}

\begin{table}
\small
\begin{tabular}{|l|c|}
\hline
\textbf{Propriedade} & \textbf{Relação} \\
\hline
Derivação no tempo & $\dfrac{d^n f}{dt^n} \xleftrightarrow{\ft} (j\omega)^n F(\omega)$ \\[4pt]
\hline
Integração no tempo & $\displaystyle\int_{-\infty}^{t} f\,d\tau \xleftrightarrow{\ft} \dfrac{F(\omega)}{j\omega} + \pi F(0)\delta(\omega)$ \\[4pt]
\hline
Dualidade & $F(t) \xleftrightarrow{\ft} 2\pi f(-\omega)$ \\
\hline
Simetria Hermitiana & $f(t)\in\mathbb{R} \;\Rightarrow\; F(-\omega) = F^*(\omega)$ \\
\hline
Parseval & $\displaystyle\int |f(t)|^2\,dt = \dfrac{1}{2\pi}\int |F(\omega)|^2\,d\omega$ \\[4pt]
\hline
\end{tabular}
\end{table}

\end{frame}

% ============================================

\begin{frame}{Resumo da Seção 3.3}

\textbf{As propriedades da Transformada de Fourier são ferramentas poderosas:}

\begin{itemize}
\item \textbf{Linearidade:} Permite decompor problemas complexos
\item \textbf{Deslocamentos:} Relacionam atrasos e modulação
\item \textbf{Escala:} Quantifica compromisso tempo-frequência
\item \textbf{Convolução:} Simplifica análise de sistemas LTI
\item \textbf{Derivação/Integração:} Resolvem equações diferenciais
\item \textbf{Dualidade:} Gera novos pares de transformadas
\item \textbf{Parseval:} Conserva energia entre domínios
\end{itemize}

\vspace{0.5cm}

\textbf{Dominando estas propriedades, você pode:}
\begin{itemize}
\item Analisar sistemas complexos sem cálculos longos
\item Entender comportamento espectral intuitivamente
\item Projetar filtros e sistemas de comunicação eficientes
\end{itemize}

\end{frame}
