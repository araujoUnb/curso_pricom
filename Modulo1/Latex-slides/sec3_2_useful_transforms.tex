% ============================================
% SEÇÃO 3.2: TRANSFORMADAS DE FUNÇÕES ÚTEIS
% ============================================

\subsection{Introdução}

\begin{frame}{Funções Fundamentais em Sistemas de Comunicação}

Certas funções aparecem frequentemente na análise de sinais e sistemas:

\begin{itemize}
\item \textbf{Função delta:} impulsos, amostragem
\item \textbf{Constantes e senoidais:} portadoras, osciladores
\item \textbf{Função degrau:} ligar/desligar, causalidade
\item \textbf{Pulsos:} transmissão de dados digitais
\item \textbf{Gaussiana:} ruído, pulsos em fibras ópticas
\end{itemize}

\vspace{0.5cm}

Conhecer as transformadas dessas funções permite:
\begin{itemize}
\item Análise rápida de sistemas complexos
\item Uso de propriedades para derivar novas transformadas
\item Compreensão do comportamento espectral
\end{itemize}

\end{frame}

% ============================================

\subsection{Função Delta de Dirac}

\begin{frame}{Função Delta de Dirac: Definição}

A \textbf{função delta de Dirac} $\delta(t)$ não é uma função no sentido clássico, mas uma \textit{distribuição} definida por suas propriedades:

\begin{enumerate}
\item $\delta(t) = 0$ para todo $t \neq 0$
\item $\int_{-\infty}^{\infty} \delta(t) dt = 1$
\item \textbf{Propriedade de amostragem:}
\[
\int_{-\infty}^{\infty} f(t) \delta(t - t_0) dt = f(t_0)
\]
\end{enumerate}

\vspace{0.3cm}

\textbf{Interpretação física:} Impulso infinitamente estreito com área unitária.

\vspace{0.3cm}

\textbf{Representação gráfica:} Seta vertical com altura indicando a "força" (área).

\end{frame}

% ============================================

\begin{frame}{Transformada da Delta de Dirac}

\textbf{Cálculo:}

\[
\FT{\delta(t)} = \int_{-\infty}^{\infty} \delta(t) e^{-j\omega t} dt
\]

Usando a propriedade de amostragem com $f(t) = e^{-j\omega t}$ e $t_0 = 0$:

\[
\FT{\delta(t)} = e^{-j\omega \cdot 0} = 1
\]

\begin{block}{Par de Transformadas}
\[
\boxed{\delta(t) \xleftrightarrow{\ft} 1}
\]
\end{block}

\textbf{Interpretação:} Um impulso no tempo contém \textbf{todas as frequências} com amplitude igual.

\textbf{Delta deslocada:} $\delta(t - t_0) \xleftrightarrow{\ft} e^{-j\omega t_0}$

\end{frame}

% ============================================

\subsection{Constante}

\begin{frame}{Transformada de uma Constante}

Queremos a transformada de $f(t) = A$ (constante para todo $t$).

\textbf{Problema:} A integral $\int_{-\infty}^{\infty} A e^{-j\omega t}\,dt$ não converge classicamente.

\textbf{Solução via dualidade:} Como $\delta(t) \xleftrightarrow{\ft} 1$, a propriedade de
dualidade ($F(t) \xleftrightarrow{\ft} 2\pi f(-\omega)$) implica diretamente:
\[
1 \xleftrightarrow{\ft} 2\pi\,\delta(-\omega) = 2\pi\,\delta(\omega)
\]

\begin{block}{Par de Transformadas}
\[
\boxed{A \xleftrightarrow{\ft} 2\pi A\,\delta(\omega)}
\]
\end{block}

\textbf{Interpretação:} Um sinal constante tem toda a sua energia em $\omega = 0$ (componente DC puro).

\end{frame}

% ============================================

\subsection{Exponencial Complexa}

\begin{frame}{Transformada da Exponencial Complexa}

Considere $f(t) = e^{j\omega_0 t}$ (senoidal complexa na frequência $\omega_0$).

Usando deslocamento em frequência (propriedade que veremos depois) ou dualidade:

\[
F(\omega) = \int_{-\infty}^{\infty} e^{j\omega_0 t} e^{-j\omega t} dt = \int_{-\infty}^{\infty} e^{-j(\omega - \omega_0) t} dt
\]

Esta integral é $2\pi\delta(\omega - \omega_0)$:

\begin{block}{Par de Transformadas}
\[
\boxed{e^{j\omega_0 t} \xleftrightarrow{\ft} 2\pi \delta(\omega - \omega_0)}
\]
\end{block}

\textbf{Interpretação:} Uma senoidal de frequência $\omega_0$ tem espectro com um único impulso em $\omega = \omega_0$.

\textbf{Significado físico:} Sinal monocromático puro.

\end{frame}

% ============================================

\begin{frame}{Transformadas de Cosseno e Seno}

Usando Euler ($\cos\omega_0 t = \tfrac{e^{j\omega_0 t}+e^{-j\omega_0 t}}{2}$) e a linearidade:

\[
\FT{\cos(\omega_0 t)}
  = \tfrac{1}{2}\cdot 2\pi\delta(\omega-\omega_0) + \tfrac{1}{2}\cdot 2\pi\delta(\omega+\omega_0)
  = \pi\bigl[\delta(\omega-\omega_0)+\delta(\omega+\omega_0)\bigr]
\]

Analogamente, com $\sin\omega_0 t = \tfrac{e^{j\omega_0 t}-e^{-j\omega_0 t}}{2j}$:

\begin{block}{Pares de Transformadas}
\[
\cos(\omega_0 t) \xleftrightarrow{\ft} \pi\bigl[\delta(\omega - \omega_0) + \delta(\omega + \omega_0)\bigr]
\]
\[
\sin(\omega_0 t) \xleftrightarrow{\ft} j\pi\bigl[\delta(\omega + \omega_0) - \delta(\omega - \omega_0)\bigr]
\]
\end{block}

\textbf{Interpretação:} Cada senoidal gera \textbf{dois impulsos} espectrais: em $+\omega_0$ e $-\omega_0$.

\end{frame}

% ============================================

\subsection{Função Degrau}

\begin{frame}{Função Degrau Unitário}

A \textbf{função degrau unitário} (Heaviside):

\[
u(t) = \begin{cases}
1 & t > 0 \\
1/2 & t = 0 \\
0 & t < 0
\end{cases}
\]

\textbf{Relação com a função sinal:}

A função sinal é definida como:
\[
\sgn(t) = \begin{cases}
1 & t > 0 \\
0 & t = 0 \\
-1 & t < 0
\end{cases}
\]

Observe que: $u(t) = \frac{1 + \sgn(t)}{2}$ e $\sgn(t) = 2u(t) - 1$

\end{frame}

% ============================================

\begin{frame}{Transformada da Função Sinal}

Para encontrar $\FT{u(t)}$, calculamos primeiro $\FT{\sgn(t)}$ via limite regulador.
Usando $e^{-at}u(t)\xleftrightarrow{\ft}\frac{1}{a+j\omega}$ e $e^{at}u(-t)\xleftrightarrow{\ft}\frac{1}{a-j\omega}$:

\[
\sgn(t) = \lim_{a\to 0^+}\!\bigl[e^{-at}u(t)-e^{at}u(-t)\bigr]
\]

\begin{align*}
\FT{\sgn(t)}
  &= \lim_{a\to 0^+}\!\left[\frac{1}{a+j\omega}-\frac{1}{a-j\omega}\right]
   = \lim_{a\to 0^+}\frac{-2j\omega}{a^2+\omega^2} \\[4pt]
  &\xrightarrow{a\to 0^+} \frac{2}{j\omega} \qquad (\omega\neq 0)
\end{align*}

\begin{block}{Transformada da Função Sinal}
\[
\sgn(t) \xleftrightarrow{\ft} \frac{2}{j\omega}
\]
\end{block}

\end{frame}

% ============================================

\begin{frame}{Transformada do Degrau Unitário}

Usando $u(t) = \frac{1 + \sgn(t)}{2}$ com os pares conhecidos
($1 \xleftrightarrow{\ft} 2\pi\delta(\omega)$ e $\sgn(t) \xleftrightarrow{\ft} \frac{2}{j\omega}$):

\[
\FT{u(t)}
  = \tfrac{1}{2}\cdot 2\pi\delta(\omega) + \tfrac{1}{2}\cdot\frac{2}{j\omega}
  = \pi\delta(\omega) + \frac{1}{j\omega}
\]

\begin{block}{Transformada do Degrau}
\[
u(t) \xleftrightarrow{\ft} \pi\delta(\omega) + \frac{1}{j\omega}
\]
\end{block}

\textbf{Interpretação:}
\begin{itemize}
\item Componente DC: $\pi\delta(\omega)$ — valor médio de $1/2$
\item Componente variável: $1/(j\omega)$ — conteúdo espectral contínuo
\end{itemize}

\end{frame}

% ============================================

\subsection{Funções de Pulso}

\begin{frame}{Pulso Retangular}

Já derivamos anteriormente:

\[
\rect(t/\tau) = \begin{cases}
1 & |t| \leq \tau/2 \\
0 & |t| > \tau/2
\end{cases}
\]

\begin{block}{Transformada}
\[
\boxed{\rect(t/\tau) \xleftrightarrow{\ft} \tau \sinc\left(\frac{\omega\tau}{2\pi}\right)}
\]
\end{block}

onde $\sinc(x) = \frac{\sin(\pi x)}{\pi x}$.

\textbf{Propriedades:}
\begin{itemize}
\item Lóbulo principal: $-2\pi/\tau < \omega < 2\pi/\tau$
\item Zeros em $\omega = \pm 2\pi n/\tau$, $n = 1, 2, 3, ...$
\item Amplitude decresce como $1/\omega$
\end{itemize}

\end{frame}

% ============================================

\begin{frame}{Função Triangular}

A \textbf{função triangular}:

\[
\tri(t/\tau) = \begin{cases}
1 - |t|/\tau & |t| \leq \tau \\
0 & |t| > \tau
\end{cases}
\]

\textbf{Observação:} $\tri(t/\tau) = \rect(t/\tau) \conv \rect(t/\tau)$ (convolução de retângulo consigo mesmo).

Pela propriedade de convolução (que veremos na próxima seção):

\[
\FT{\tri(t/\tau)} = \FT{\rect(t/\tau)} \cdot \FT{\rect(t/\tau)} = \left[\tau \sinc\left(\frac{\omega\tau}{2\pi}\right)\right]^2
\]

\begin{block}{Transformada}
\[
\boxed{\tri(t/\tau) \xleftrightarrow{\ft} \tau \sinc^2\left(\frac{\omega\tau}{2\pi}\right)}
\]
\end{block}

\textbf{Nota:} O espectro decai mais rápido ($\propto 1/\omega^2$) que o retângulo.

\end{frame}

% ============================================

\subsection{Função Gaussiana}

\begin{frame}{Pulso Gaussiano}

A \textbf{função Gaussiana}:

\[
f(t) = e^{-\alpha t^2}, \quad \alpha > 0
\]

\textbf{Cálculo da transformada:}

\[
F(\omega) = \int_{-\infty}^{\infty} e^{-\alpha t^2} e^{-j\omega t} dt
\]

Completando o quadrado no expoente:
\[
-\alpha t^2 - j\omega t = -\alpha\left(t + \frac{j\omega}{2\alpha}\right)^2 - \frac{\omega^2}{4\alpha}
\]

\[
F(\omega) = e^{-\omega^2/(4\alpha)} \int_{-\infty}^{\infty} e^{-\alpha\left(t + \frac{j\omega}{2\alpha}\right)^2} dt
\]

\end{frame}

% ============================================

\begin{frame}{Pulso Gaussiano (continuação)}

Com $u = t + j\omega/(2\alpha)$, $du = dt$, e a integral gaussiana $\int e^{-\alpha u^2}du = \sqrt{\pi/\alpha}$:

\begin{align*}
F(\omega)
  &= e^{-\omega^2/(4\alpha)} \int_{-\infty}^{\infty} e^{-\alpha u^2}\,du
   = e^{-\omega^2/(4\alpha)}\cdot\sqrt{\tfrac{\pi}{\alpha}}
\end{align*}

\begin{block}{Transformada da Gaussiana}
\[
e^{-\alpha t^2} \xleftrightarrow{\ft} \sqrt{\frac{\pi}{\alpha}}\,e^{-\omega^2/(4\alpha)}
\]
\end{block}

\textbf{Propriedade notável:} Uma Gaussiana transforma-se em outra Gaussiana!

\textbf{Princípio de incerteza:} A Gaussiana é o sinal com mínimo produto $\Delta t \cdot \Delta\omega = 1/2$.

\end{frame}

% ============================================

\subsection{Trem de Impulsos}

\begin{frame}{Trem de Impulsos Periódico}

Um \textbf{trem de impulsos} (impulse train) com período $T$:

\[
f(t) = \sum_{n=-\infty}^{\infty} \delta(t - nT)
\]

Este sinal é periódico ($\omega_0 = 2\pi/T$) e expande em série de Fourier. Os coeficientes valem
\[
c_n = \frac{1}{T}\!\int_{-T/2}^{T/2}\!\delta(t)\,e^{-jn\omega_0 t}\,dt = \frac{1}{T}
\]
logo:
\[
f(t) = \frac{1}{T} \sum_{n=-\infty}^{\infty} e^{jn\omega_0 t}
\]

\textbf{Interpretação:} Todos os harmônicos têm a mesma amplitude $1/T$ — espectro plano!

\end{frame}

% ============================================

\begin{frame}{Transformada do Trem de Impulsos}

Usando a linearidade e $e^{jn\omega_0 t} \xleftrightarrow{\ft} 2\pi\delta(\omega - n\omega_0)$:

\[
F(\omega) = \frac{1}{T}\sum_{n} 2\pi\delta(\omega - n\omega_0)
           = \frac{2\pi}{T}\sum_{n=-\infty}^{\infty}\delta(\omega - n\omega_0)
\]

\begin{block}{Par de Transformadas: Trem de Impulsos}
\[
\sum_{n=-\infty}^{\infty} \delta(t - nT)
\;\xleftrightarrow{\ft}\;
\frac{2\pi}{T} \sum_{n=-\infty}^{\infty} \delta\!\left(\omega - \frac{2\pi n}{T}\right)
\]
\end{block}

\begin{itemize}
\item \textbf{Simetria perfeita:} impulsos periódicos no tempo $\leftrightarrow$ impulsos periódicos na frequência
\item \textbf{Separação espectral:} $\Delta\omega = 2\pi/T$ (inversamente proporcional a $T$)
\item \textbf{Aplicação fundamental:} base matemática do teorema de amostragem
\end{itemize}

\end{frame}

% ============================================

\begin{frame}{Tabela Resumo: Transformadas de Funções Úteis}

\begin{table}
\small
\begin{tabular}{|c|c|}
\hline
\textbf{Sinal} $f(t)$ & \textbf{Transformada} $F(\omega)$ \\
\hline
$\delta(t)$ & $1$ \\
\hline
$1$ & $2\pi\delta(\omega)$ \\
\hline
$e^{j\omega_0 t}$ & $2\pi\delta(\omega - \omega_0)$ \\
\hline
$\cos(\omega_0 t)$ & $\pi[\delta(\omega - \omega_0) + \delta(\omega + \omega_0)]$ \\
\hline
$\sin(\omega_0 t)$ & $j\pi[\delta(\omega + \omega_0) - \delta(\omega - \omega_0)]$ \\
\hline
$u(t)$ & $\pi\delta(\omega) + \frac{1}{j\omega}$ \\
\hline
$\sgn(t)$ & $\frac{2}{j\omega}$ \\
\hline
$e^{-at}u(t)$, $a > 0$ & $\frac{1}{a+j\omega}$ \\
\hline
$e^{-a|t|}$, $a > 0$ & $\frac{2a}{a^2+\omega^2}$ \\
\hline
\end{tabular}
\end{table}

\end{frame}

% ============================================

\begin{frame}{Tabela Resumo (continuação)}

\begin{table}
\small
\begin{tabular}{|c|c|}
\hline
\textbf{Sinal} $f(t)$ & \textbf{Transformada} $F(\omega)$ \\
\hline
$\rect(t/\tau)$ & $\tau \sinc(\omega\tau/2\pi)$ \\
\hline
$\tri(t/\tau)$ & $\tau \sinc^2(\omega\tau/2\pi)$ \\
\hline
$e^{-\alpha t^2}$ & $\sqrt{\pi/\alpha} \, e^{-\omega^2/(4\alpha)}$ \\
\hline
$\sum_{n=-\infty}^{\infty} \delta(t - nT)$ & $\frac{2\pi}{T} \sum_{n=-\infty}^{\infty} \delta(\omega - 2\pi n/T)$ \\
\hline
\end{tabular}
\end{table}

\vspace{0.5cm}

\textbf{Essas transformadas são fundamentais!}

Memorize os pares mais importantes e use propriedades (próxima seção) para derivar outros.

\end{frame}

% ============================================

\begin{frame}{Observações e Padrões}

\textbf{Padrões observados:}

\begin{enumerate}
\item \textbf{Dualidade tempo-frequência:}
   \begin{itemize}
   \item Impulso no tempo $\leftrightarrow$ constante na frequência
   \item Constante no tempo $\leftrightarrow$ impulso na frequência
   \item Impulsos periódicos $\leftrightarrow$ impulsos periódicos
   \end{itemize}

\item \textbf{Decaimento:}
   \begin{itemize}
   \item Decaimento exponencial $\leftrightarrow$ Lorentziana
   \item Gaussiana $\leftrightarrow$ Gaussiana
   \end{itemize}

\item \textbf{Suavidade:}
   \begin{itemize}
   \item Descontinuidades $\rightarrow$ decaimento lento do espectro ($1/\omega$)
   \item Continuidade $\rightarrow$ decaimento mais rápido ($1/\omega^2$ ou exponencial)
   \end{itemize}
\end{enumerate}

\end{frame}
