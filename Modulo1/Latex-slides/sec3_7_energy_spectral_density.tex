% ============================================
% SEÇÃO 3.7: ENERGIA E DENSIDADE ESPECTRAL DE ENERGIA
% ============================================

\subsection{Energia de Sinais}

\begin{frame}{Classificação de Sinais por Energia e Potência}

Sinais podem ser classificados em duas categorias:

\begin{enumerate}
\item \textbf{Sinais de Energia:}
   \begin{itemize}
   \item Energia finita: $0 < E < \infty$
   \item Potência média: $P = 0$
   \item Exemplos: pulsos, sinais transitórios, sinais causais decrescentes
   \end{itemize}

\item \textbf{Sinais de Potência:}
   \begin{itemize}
   \item Energia infinita: $E = \infty$
   \item Potência média finita: $0 < P < \infty$
   \item Exemplos: sinais periódicos, constantes, aleatórios estacionários
   \end{itemize}
\end{enumerate}

\vspace{0.3cm}

\textbf{Nota:} Alguns sinais (como $t$ ou $e^t$) não se enquadram em nenhuma categoria: energia e potência infinitas.

\textbf{Foco desta seção:} Sinais de energia.

\end{frame}

% ============================================

\begin{frame}{Energia de um Sinal no Domínio do Tempo}

A \textbf{energia} de um sinal $f(t)$ é definida como:

\[
E = \int_{-\infty}^{\infty} |f(t)|^2 dt
\]

\textbf{Interpretação física:}

\begin{itemize}
\item Para sinal de tensão $v(t)$ em resistor de 1 $\Omega$: energia dissipada
\item $|f(t)|^2$ é a potência instantânea
\item Integral acumula energia total ao longo do tempo
\end{itemize}

\vspace{0.3cm}

\textbf{Unidades:}

\begin{itemize}
\item Se $f(t)$ é tensão em volts: energia em joules
\item Se $f(t)$ é normalizado: energia adimensional
\end{itemize}

\vspace{0.3cm}

\textbf{Para sinais complexos:} $|f(t)|^2 = f(t) \cdot f^*(t)$ onde $f^*(t)$ é o conjugado complexo.

\end{frame}

% ============================================

\begin{frame}{Exemplos: Energia de Sinais}

\textbf{Exemplo 1: Pulso retangular}

$f(t) = A\rect(t/\tau)$

\[
E = \int_{-\tau/2}^{\tau/2} A^2 dt = A^2\tau
\]

Energia proporcional à amplitude ao quadrado e à duração.

\vspace{0.3cm}

\textbf{Exemplo 2: Exponencial decrescente}

$f(t) = e^{-at}u(t)$, $a > 0$

\[
E = \int_{0}^{\infty} e^{-2at} dt = \left[-\frac{e^{-2at}}{2a}\right]_0^{\infty} = \frac{1}{2a}
\]

Energia inversamente proporcional à taxa de decaimento.

\vspace{0.3cm}

\textbf{Exemplo 3: Senoide}

$f(t) = A\cos(\omega_0 t)$ (para todo $t$)

\[
E = \int_{-\infty}^{\infty} A^2\cos^2(\omega_0 t) dt = \infty
\]

Sinal periódico tem energia infinita (é sinal de potência).

\end{frame}

% ============================================

\subsection{Teorema de Parseval}

\begin{frame}{Energia no Domínio da Frequência}

\textbf{Teorema de Parseval} relaciona energia nos domínios do tempo e frequência.

\begin{block}{Teorema de Parseval}
Se $f(t) \xleftrightarrow{\ft} F(\omega)$, então:
\[
E = \int_{-\infty}^{\infty} |f(t)|^2 dt = \frac{1}{2\pi} \int_{-\infty}^{\infty} |F(\omega)|^2 d\omega
\]
\end{block}

\textbf{Significado:}

\begin{itemize}
\item A energia total é a mesma calculada no tempo ou na frequência
\item A energia é conservada entre os domínios
\item $|F(\omega)|^2/(2\pi)$ representa densidade de energia por unidade de frequência
\end{itemize}

\vspace{0.3cm}

\textbf{Analogia:} Como a energia total de um sistema físico permanece constante em diferentes sistemas de coordenadas.

\end{frame}

% ============================================

\begin{frame}{Demonstração do Teorema de Parseval (revisão)}

Começamos com a definição de energia:

\[
E = \int_{-\infty}^{\infty} |f(t)|^2 dt = \int_{-\infty}^{\infty} f(t) f^*(t) dt
\]

Substituindo a transformada inversa para $f^*(t)$:

\[
f^*(t) = \left[\frac{1}{2\pi} \int_{-\infty}^{\infty} F(\omega) e^{j\omega t} d\omega\right]^* = \frac{1}{2\pi} \int_{-\infty}^{\infty} F^*(\omega) e^{-j\omega t} d\omega
\]

Portanto:

\[
E = \int_{-\infty}^{\infty} f(t) \left[\frac{1}{2\pi} \int_{-\infty}^{\infty} F^*(\omega) e^{-j\omega t} d\omega\right] dt
\]

Trocando a ordem de integração:

\[
E = \frac{1}{2\pi} \int_{-\infty}^{\infty} F^*(\omega) \left[\int_{-\infty}^{\infty} f(t) e^{-j\omega t} dt\right] d\omega
\]

\end{frame}

% ============================================

\begin{frame}{Demonstração do Teorema de Parseval (conclusão)}

A integral interna é justamente $F(\omega)$:

\[
E = \frac{1}{2\pi} \int_{-\infty}^{\infty} F^*(\omega) F(\omega) d\omega
\]

\[
= \frac{1}{2\pi} \int_{-\infty}^{\infty} |F(\omega)|^2 d\omega
\]

Portanto:

\[
\boxed{\int_{-\infty}^{\infty} |f(t)|^2 dt = \frac{1}{2\pi} \int_{-\infty}^{\infty} |F(\omega)|^2 d\omega}
\]

\textbf{Conclusão:} A energia pode ser calculada em qualquer domínio, com resultado idêntico.

\textbf{Importância prática:} Às vezes é mais fácil calcular energia na frequência que no tempo.

\end{frame}

% ============================================

\subsection{Densidade Espectral de Energia}

\begin{frame}{Definição de Densidade Espectral de Energia}

A \textbf{Densidade Espectral de Energia (DEE)} ou \textit{Energy Spectral Density (ESD)} é definida como:

\begin{block}{Densidade Espectral de Energia}
\[
\Psi(\omega) = |F(\omega)|^2
\]
\end{block}

\textbf{Unidades:} Energia por Hz (ou por rad/s, dependendo se usa $f$ ou $\omega$).

\textbf{Interpretação:}

\begin{itemize}
\item $\Psi(\omega)$ indica quanto de energia está concentrado em cada frequência
\item Áreas sob $\Psi(\omega)$ representam energia em bandas de frequência
\item Pelo Teorema de Parseval:
\[
E = \frac{1}{2\pi} \int_{-\infty}^{\infty} \Psi(\omega) d\omega
\]
\end{itemize}

\textbf{Propriedade:} Para sinais reais, $\Psi(\omega)$ é par: $\Psi(-\omega) = \Psi(\omega)$

\end{frame}

% ============================================

\begin{frame}{Exemplo: DEE de Pulso Retangular}

\textbf{Sinal:} $f(t) = A\rect(t/\tau)$

Já sabemos que:
\[
F(\omega) = A\tau \sinc\left(\frac{\omega\tau}{2\pi}\right)
\]

\textbf{Densidade espectral de energia:}

\[
\Psi(\omega) = |F(\omega)|^2 = A^2\tau^2 \sinc^2\left(\frac{\omega\tau}{2\pi}\right)
\]

\textbf{Características:}

\begin{itemize}
\item Máximo em $\omega = 0$: $\Psi(0) = A^2\tau^2$
\item Zeros em $\omega = \pm 2\pi n/\tau$ para $n = 1, 2, 3, ...$
\item Maior parte da energia no lóbulo principal: $|\omega| < 2\pi/\tau$
\item Decai como $1/\omega^2$ para grandes $\omega$
\end{itemize}

\textbf{Verificação do Teorema de Parseval:} A integral de $\Psi(\omega)/(2\pi)$ deve dar $A^2\tau$ (energia no tempo).

\end{frame}

% ============================================

\begin{frame}{Exemplo: DEE de Exponencial Decrescente}

\textbf{Sinal:} $f(t) = e^{-at}u(t)$, $a > 0$

Transformada:
\[
F(\omega) = \frac{1}{a + j\omega}
\]

\textbf{Densidade espectral de energia:}

\[
\Psi(\omega) = |F(\omega)|^2 = \frac{1}{|a + j\omega|^2} = \frac{1}{a^2 + \omega^2}
\]

\textbf{Características:}

\begin{itemize}
\item Forma Lorentziana
\item Máximo em $\omega = 0$: $\Psi(0) = 1/a^2$
\item Em $\omega = \pm a$: $\Psi(a) = 1/(2a^2)$ (metade do máximo)
\item Decai como $1/\omega^2$ para $\omega \to \infty$
\end{itemize}

\textbf{Energia total:}

\[
E = \frac{1}{2\pi} \int_{-\infty}^{\infty} \frac{1}{a^2 + \omega^2} d\omega = \frac{1}{2\pi} \cdot \frac{2\pi}{2a} = \frac{1}{2a}
\]

Confirma cálculo no tempo!

\end{frame}

% ============================================

\subsection{Largura de Banda de Energia}

\begin{frame}{Conceito de Largura de Banda}

A \textbf{largura de banda} de um sinal quantifica a faixa de frequências que contém sua energia significativa.

\textbf{Definições comuns:}

\begin{enumerate}
\item \textbf{Largura de banda absoluta:}
   
   Menor intervalo $[-W, W]$ onde $F(\omega) \neq 0$
   
   (pode ser infinita)

\item \textbf{Largura de banda de 3 dB:}
   
   Faixa onde $\Psi(\omega) \geq \Psi_{\max}/2$

\item \textbf{Largura de banda de ruído:}
   
   Largura de um retângulo equivalente de mesma área e altura máxima:
   \[
   B_n = \frac{1}{\Psi(0)} \int_{-\infty}^{\infty} \Psi(\omega) d\omega = \frac{2\pi E}{\Psi(0)}
   \]

\item \textbf{Largura de banda de X\%:}
   
   Menor faixa $[-W_x, W_x]$ contendo X\% da energia total
\end{enumerate}

\end{frame}

% ============================================

\begin{frame}{Cálculo de Largura de Banda de X\%}

\textbf{Definição:} Encontrar $W_x$ tal que:

\[
\frac{1}{2\pi} \int_{-W_x}^{W_x} \Psi(\omega) d\omega = \alpha E
\]

onde $\alpha$ é a fração desejada (ex: 0.90 para 90\%, 0.99 para 99\%).

\vspace{0.3cm}

\textbf{Exemplo: Pulso retangular}

Para $f(t) = \rect(t/\tau)$ com $\Psi(\omega) = \tau^2 \sinc^2(\omega\tau/2\pi)$:

\begin{itemize}
\item Lóbulo principal ($|\omega| < 2\pi/\tau$) contém $\approx 90\%$ da energia
\item $W_{90\%} \approx 2\pi/\tau$ ou $B_{90\%} \approx 1/\tau$ em Hz
\end{itemize}

\textbf{Princípio geral:}

\[
\text{Duração no tempo} \times \text{Largura de banda} \approx \text{constante}
\]

Pulso mais curto $\rightarrow$ espectro mais largo.

\end{frame}

% ============================================

\subsection{Relação de Incerteza}

\begin{frame}{Desigualdade de Heisenberg-Gabor}

\textbf{Duração efetiva} de um sinal:

\[
\Delta t = \sqrt{\frac{\int_{-\infty}^{\infty} t^2 |f(t)|^2 dt}{\int_{-\infty}^{\infty} |f(t)|^2 dt}}
\]

\textbf{Largura de banda efetiva:}

\[
\Delta \omega = \sqrt{\frac{\int_{-\infty}^{\infty} \omega^2 |F(\omega)|^2 d\omega}{\int_{-\infty}^{\infty} |F(\omega)|^2 d\omega}}
\]

\begin{block}{Desigualdade de Heisenberg-Gabor}
\[
\Delta t \cdot \Delta \omega \geq \frac{1}{2}
\]
\end{block}

\textbf{Significado:}

\begin{itemize}
\item Produto tempo-frequência tem limite inferior
\item Não podemos ter sinal arbitrariamente concentrado em ambos os domínios
\item Compromisso fundamental: localização no tempo $\leftrightarrow$ localização em frequência
\end{itemize}

\end{frame}

% ============================================

\begin{frame}{Sinal Ótimo: Gaussiana}

O \textbf{pulso Gaussiano} atinge a igualdade na desigualdade:

\[
f(t) = e^{-\alpha t^2} \quad \xleftrightarrow{\ft} \quad F(\omega) = \sqrt{\frac{\pi}{\alpha}} e^{-\omega^2/(4\alpha)}
\]

Para este sinal:

\[
\Delta t \cdot \Delta \omega = \frac{1}{2}
\]

\textbf{Propriedades da Gaussiana:}

\begin{itemize}
\item Gaussiana no tempo $\leftrightarrow$ Gaussiana na frequência
\item Melhor localização simultânea tempo-frequência
\item Mínima incerteza
\item Usada em comunicações ópticas e rádio
\end{itemize}

\vspace{0.3cm}

\textbf{Todos os outros sinais:} $\Delta t \cdot \Delta \omega > 1/2$

\end{frame}

% ============================================

\subsection{Energia na Saída de Sistemas LTI}

\begin{frame}{Energia através de Sistema LTI}

\textbf{Sistema:} Entrada $x(t)$ com transformada $X(\omega)$, saída $y(t)$.

\textbf{Relação:} $Y(\omega) = H(\omega) X(\omega)$

\textbf{Energia da entrada:}

\[
E_x = \frac{1}{2\pi} \int_{-\infty}^{\infty} |X(\omega)|^2 d\omega
\]

\textbf{Energia da saída:}

\[
E_y = \frac{1}{2\pi} \int_{-\infty}^{\infty} |Y(\omega)|^2 d\omega = \frac{1}{2\pi} \int_{-\infty}^{\infty} |H(\omega)|^2 |X(\omega)|^2 d\omega
\]

\begin{block}{Relação Entrada-Saída}
\[
\Psi_y(\omega) = |H(\omega)|^2 \Psi_x(\omega)
\]
\end{block}

\textbf{Interpretação:} A DEE da saída é a DEE da entrada multiplicada pelo quadrado da magnitude da resposta em frequência.

\end{frame}

% ============================================

\begin{frame}{Exemplo: Pulso através de Filtro RC}

\textbf{Entrada:} Pulso $x(t) = \delta(t)$ (impulso unitário)

$X(\omega) = 1$, $\Psi_x(\omega) = 1$

\textbf{Sistema:} Filtro RC passa-baixas

\[
H(\omega) = \frac{1}{1 + j\omega RC}
\]

\textbf{DEE da saída:}

\[
\Psi_y(\omega) = |H(\omega)|^2 = \frac{1}{1 + (\omega RC)^2}
\]

\textbf{Energia da saída:}

\[
E_y = \frac{1}{2\pi} \int_{-\infty}^{\infty} \frac{1}{1 + (\omega RC)^2} d\omega
\]

Com $\omega_c = 1/(RC)$:

\[
E_y = \frac{1}{2\pi} \cdot \frac{2\pi}{\omega_c \cdot 2} = \frac{1}{2\omega_c} = \frac{RC}{2}
\]

\end{frame}

% ============================================

\begin{frame}{Energia em Banda Limitada}

\textbf{Problema:} Qual fração da energia passa por um filtro passa-baixas ideal?

\textbf{Filtro:} $H(\omega) = \rect(\omega/2W)$ com largura de banda $W$

\textbf{Energia transmitida:}

\[
E_{pass} = \frac{1}{2\pi} \int_{-W}^{W} \Psi_x(\omega) d\omega
\]

\textbf{Fração da energia:}

\[
\eta = \frac{E_{pass}}{E_{total}} = \frac{\int_{-W}^{W} \Psi_x(\omega) d\omega}{\int_{-\infty}^{\infty} \Psi_x(\omega) d\omega}
\]

\vspace{0.3cm}

\textbf{Exemplo: Exponencial} $e^{-a|t|}$ com $\Psi(\omega) = 4a^2/(a^2+\omega^2)^2$

Para $W = 3a$: $\eta \approx 0.95$ (95\% da energia dentro de $[-3a, 3a]$)

\textbf{Aplicação:} Projeto de sistemas com banda limitada.

\end{frame}

% ============================================

\subsection{Autocorrelação e DEE}

\begin{frame}{Função de Autocorrelação}

A \textbf{função de autocorrelação} de um sinal de energia $f(t)$ é:

\[
R(\tau) = \int_{-\infty}^{\infty} f(t) f^*(t - \tau) dt
\]

\textbf{Propriedades:}

\begin{itemize}
\item $R(0) = E$ (energia total)
\item $R(\tau) = R^*(-\tau)$ (simetria Hermitiana)
\item Para sinais reais: $R(\tau) = R(-\tau)$ (par)
\item $|R(\tau)| \leq R(0) = E$
\end{itemize}

\vspace{0.3cm}

\textbf{Interpretação:} Mede similaridade do sinal com versão deslocada de si mesmo.

\begin{block}{Teorema de Wiener-Khinchin (Energia)}
\[
R(\tau) \xleftrightarrow{\ft} \Psi(\omega) = |F(\omega)|^2
\]
\end{block}

A autocorrelação e a DEE são pares de Fourier!

\end{frame}

% ============================================

\begin{frame}{Resumo da Seção 3.7}

\textbf{Conceitos fundamentais:}

\begin{itemize}
\item \textbf{Energia no tempo:} $E = \int |f(t)|^2 dt$
\item \textbf{Energia na frequência:} $E = \frac{1}{2\pi}\int |F(\omega)|^2 d\omega$
\item \textbf{Teorema de Parseval:} Energia conservada entre domínios
\end{itemize}

\vspace{0.3cm}

\textbf{Densidade Espectral de Energia:}

\begin{itemize}
\item $\Psi(\omega) = |F(\omega)|^2$
\item Indica distribuição de energia em frequência
\item Em sistemas LTI: $\Psi_y(\omega) = |H(\omega)|^2 \Psi_x(\omega)$
\end{itemize}

\vspace{0.3cm}

\textbf{Princípios importantes:}

\begin{itemize}
\item Relação de incerteza: $\Delta t \cdot \Delta \omega \geq 1/2$
\item Gaussiana é ótima (atinge igualdade)
\item Largura de banda relacionada com duração temporal
\end{itemize}

\textbf{Próximo:} Sinais de potência e densidade espectral de potência.

\end{frame}
