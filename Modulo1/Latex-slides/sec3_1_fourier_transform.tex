% ============================================
% SEÇÃO 3.1: TRANSFORMADA DE FOURIER DE SINAIS
% ============================================

\subsection{Motivação}

\begin{frame}{Por que a Transformada de Fourier?}

A análise de sistemas de comunicação requer ferramentas para:

\begin{itemize}
\item Entender como os sinais se comportam em diferentes frequências
\item Projetar filtros e sistemas de transmissão
\item Calcular larguras de banda necessárias
\item Analisar distorção e interferência
\end{itemize}

\vspace{0.5cm}

A \textbf{Transformada de Fourier} é a ferramenta fundamental que permite:
\begin{itemize}
\item Passar do domínio do tempo para o domínio da frequência
\item Analisar o conteúdo espectral de sinais
\item Simplificar a análise de sistemas lineares
\end{itemize}

\end{frame}

% ============================================

\subsection{Derivação da Transformada de Fourier}

\begin{frame}{Série de Fourier para Sinais Periódicos}

Para um sinal periódico $f(t)$ com período $T_0$ e frequência fundamental $\omega_0 = 2\pi/T_0$:

\[
f(t) = \sum_{n=-\infty}^{\infty} c_n e^{jn\omega_0 t}
\]

onde os coeficientes são:

\[
c_n = \frac{1}{T_0} \int_{-T_0/2}^{T_0/2} f(t) e^{-jn\omega_0 t} dt
\]

\vspace{0.3cm}

\textbf{Ideia:} Representar qualquer sinal periódico como soma de exponenciais complexas.

\vspace{0.3cm}

\textbf{E sinais não-periódicos?} Podemos pensar neles como sinais periódicos com $T_0 \to \infty$.

\end{frame}

% ============================================

\begin{frame}{De Série para Transformada: Passo 1}

Substituindo $c_n$ na série de Fourier:

\[
f(t) = \sum_{n=-\infty}^{\infty} \left[ \frac{1}{T_0} \int_{-T_0/2}^{T_0/2} f(\tau) e^{-jn\omega_0 \tau} d\tau \right] e^{jn\omega_0 t}
\]

Reorganizando:

\[
f(t) = \sum_{n=-\infty}^{\infty} \left[ \int_{-T_0/2}^{T_0/2} f(\tau) e^{-jn\omega_0 \tau} d\tau \right] \frac{e^{jn\omega_0 t}}{T_0}
\]

Como $\omega_0 = 2\pi/T_0$, temos $1/T_0 = \omega_0/(2\pi)$:

\[
f(t) = \sum_{n=-\infty}^{\infty} \left[ \int_{-T_0/2}^{T_0/2} f(\tau) e^{-jn\omega_0 \tau} d\tau \right] \frac{\omega_0}{2\pi} e^{jn\omega_0 t}
\]

\end{frame}

% ============================================

\begin{frame}{De Série para Transformada: Passo 2}

Quando $T_0 \to \infty$:

\begin{itemize}
\item O espaçamento entre harmônicas $\omega_0 \to 0$ (espectro contínuo)
\item A variável discreta $n\omega_0$ torna-se contínua: $\omega$
\item O intervalo de integração $[-T_0/2, T_0/2] \to [-\infty, \infty]$
\item A soma $\sum (\cdots) \omega_0$ torna-se integral $\int (\cdots) d\omega$
\end{itemize}

\vspace{0.5cm}

Definindo a \textbf{Transformada de Fourier}:

\[
F(\omega) = \lim_{T_0 \to \infty} \int_{-T_0/2}^{T_0/2} f(t) e^{-j\omega t} dt = \int_{-\infty}^{\infty} f(t) e^{-j\omega t} dt
\]

\end{frame}

% ============================================

\begin{frame}{Par de Transformadas de Fourier}

\begin{block}{Definições}
\textbf{Transformada de Fourier (direta):}
\[
F(\omega) = \FT{f(t)} = \int_{-\infty}^{\infty} f(t) e^{-j\omega t} dt
\]

\textbf{Transformada Inversa de Fourier:}
\[
f(t) = \IFT{F(\omega)} = \frac{1}{2\pi} \int_{-\infty}^{\infty} F(\omega) e^{j\omega t} d\omega
\]

Notação: $f(t) \xleftrightarrow{\ft} F(\omega)$
\end{block}

\vspace{0.3cm}

\textbf{Interpretação:} $F(\omega)$ representa a "quantidade" de cada frequência $\omega$ presente no sinal $f(t)$.

\end{frame}

% ============================================

\begin{frame}{Condições de Existência}

A transformada de Fourier $F(\omega)$ existe se $f(t)$ satisfaz as \textbf{condições de Dirichlet}:

\begin{enumerate}
\item $f(t)$ é absolutamente integrável:
\[
\int_{-\infty}^{\infty} |f(t)| dt < \infty
\]

\item $f(t)$ tem número finito de máximos e mínimos em qualquer intervalo finito

\item $f(t)$ tem número finito de descontinuidades finitas em qualquer intervalo finito
\end{enumerate}

\vspace{0.3cm}

\textbf{Observação:} Sinais de energia finita sempre têm transformada de Fourier. Para sinais de potência (como senoidais), usa-se a função delta de Dirac.

\end{frame}

% ============================================

\subsection{Espectro de Magnitude e Fase}

\begin{frame}{Representação Espectral}

A transformada $F(\omega)$ é geralmente complexa:

\[
F(\omega) = |F(\omega)| e^{j\phi(\omega)} = \Real\{F(\omega)\} + j\Imag\{F(\omega)\}
\]

onde:

\begin{itemize}
\item \textbf{Espectro de magnitude:} $|F(\omega)| = \sqrt{\Real^2\{F(\omega)\} + \Imag^2\{F(\omega)\}}$
\item \textbf{Espectro de fase:} $\phi(\omega) = \arctan\left(\frac{\Imag\{F(\omega)\}}{\Real\{F(\omega)\}}\right)$
\end{itemize}

\vspace{0.5cm}

O espectro completo requer ambas informações: magnitude \textbf{e} fase.

Para sinais reais, temos simetrias úteis:
\begin{itemize}
\item $|F(-\omega)| = |F(\omega)|$ (magnitude par)
\item $\phi(-\omega) = -\phi(\omega)$ (fase ímpar)
\end{itemize}

\end{frame}

% ============================================

\subsection{Exemplos}

\begin{frame}{Exemplo 1: Pulso Retangular}

Considere um pulso retangular de duração $\tau$ e amplitude $A$:

\[
f(t) = \begin{cases}
A & |t| \leq \tau/2 \\
0 & |t| > \tau/2
\end{cases} = A \cdot \rect(t/\tau)
\]

\textbf{Cálculo da transformada:}

\[
F(\omega) = \int_{-\infty}^{\infty} f(t) e^{-j\omega t} dt = \int_{-\tau/2}^{\tau/2} A e^{-j\omega t} dt
\]

\[
= A \left[ \frac{e^{-j\omega t}}{-j\omega} \right]_{-\tau/2}^{\tau/2} = A \frac{e^{-j\omega\tau/2} - e^{j\omega\tau/2}}{-j\omega}
\]

\end{frame}

% ============================================

\begin{frame}{Exemplo 1: Pulso Retangular (continuação)}

Usando a identidade de Euler: $\sin(\theta) = \frac{e^{j\theta} - e^{-j\theta}}{2j}$

\[
F(\omega) = A \frac{e^{j\omega\tau/2} - e^{-j\omega\tau/2}}{j\omega} = A \frac{2\sin(\omega\tau/2)}{\omega}
\]

\[
= A\tau \frac{\sin(\omega\tau/2)}{\omega\tau/2}
\]

Definindo a \textbf{função sinc}: $\sinc(x) = \frac{\sin(\pi x)}{\pi x}$

Obtemos:

\[
F(\omega) = A\tau \sinc\left(\frac{\omega\tau}{2\pi}\right)
\]

\textbf{Conclusão:} Pulso retangular no tempo $\leftrightarrow$ função sinc na frequência.

\end{frame}

% ============================================

\begin{frame}{Exemplo 1: Análise do Resultado}

\begin{block}{Par de Transformadas: Pulso Retangular}
\[
A\,\rect\!\left(\frac{t}{\tau}\right) \xleftrightarrow{\ft} A\tau\,\sinc\!\left(\frac{\omega\tau}{2\pi}\right)
\]
\end{block}

\begin{columns}[T]
\column{0.42\textwidth}
\textbf{Observações:}
\begin{itemize}
\item Primeiro zero: $\omega = \pm 2\pi/\tau$
\item Largura de banda: $B \approx 1/\tau$
\item Pulso \textbf{estreito} $\Rightarrow$ espectro \textbf{largo}
\item $F(0) = A\tau$ (área total do pulso)
\item Espectro real pois $f(t)$ é par
\end{itemize}

\column{0.58\textwidth}
\vspace{-0.3cm}
\begin{center}
\includegraphics[width=\linewidth, height=0.55\textheight, keepaspectratio]{figures/cap3/rect_fourier}
\end{center}
\end{columns}

\end{frame}

% ============================================

\begin{frame}{Exemplo 2: Exponencial Decrescente}

Considere um sinal exponencial causal:

\[
f(t) = \begin{cases}
e^{-at} & t \geq 0 \\
0 & t < 0
\end{cases} = e^{-at} u(t), \quad a > 0
\]

onde $u(t)$ é a função degrau unitário.

\textbf{Cálculo da transformada:}

\[
F(\omega) = \int_{-\infty}^{\infty} f(t) e^{-j\omega t} dt = \int_{0}^{\infty} e^{-at} e^{-j\omega t} dt
\]

\[
= \int_{0}^{\infty} e^{-(a+j\omega)t} dt = \left[ \frac{e^{-(a+j\omega)t}}{-(a+j\omega)} \right]_{0}^{\infty}
\]

\end{frame}

% ============================================

\begin{frame}{Exemplo 2: Exponencial Decrescente (continuação)}

Como $a > 0$, temos $e^{-(a+j\omega)t} \to 0$ quando $t \to \infty$:

\[
F(\omega) = 0 - \frac{1}{-(a+j\omega)} = \frac{1}{a+j\omega}
\]

Racionalizando (multiplicando por conjugado):

\[
F(\omega) = \frac{1}{a+j\omega} \cdot \frac{a-j\omega}{a-j\omega} = \frac{a-j\omega}{a^2+\omega^2}
\]

\[
\boxed{F(\omega) = \frac{a}{a^2+\omega^2} - j\frac{\omega}{a^2+\omega^2}}
\]

\textbf{Espectro de magnitude:}
\[
|F(\omega)| = \frac{1}{\sqrt{a^2+\omega^2}}
\]

\end{frame}

% ============================================

\begin{frame}{Exemplo 2: Análise do Resultado}

\[
\boxed{f(t) = e^{-at}u(t) \quad \xleftrightarrow{\ft} \quad F(\omega) = \frac{1}{a+j\omega}}
\]

\textbf{Características do espectro:}

\begin{itemize}
\item Magnitude: $|F(\omega)| = 1/\sqrt{a^2+\omega^2}$ (forma Lorentziana)
\item Fase: $\phi(\omega) = -\arctan(\omega/a)$
\item Em $\omega = 0$: $|F(0)| = 1/a$ (valor máximo)
\item Em $\omega = \pm a$: $|F(\omega)| = 1/(a\sqrt{2})$ (redução de 3 dB)
\item Largura de banda: $B \approx a$ (relacionada à constante de decaimento)
\end{itemize}

\vspace{0.3cm}

\textbf{Interpretação física:} Decaimento rápido no tempo ($a$ grande) $\rightarrow$ espectro largo na frequência.

\end{frame}

% ============================================

\begin{frame}{Exemplo 2: Visualização do Espectro}

\begin{center}
\includegraphics[width=\figFull, height=0.72\textheight, keepaspectratio]{figures/cap3/exponential_fourier}
\end{center}

\vspace{-0.3cm}
\begin{itemize}
\item Decaimento mais rápido no tempo ($a$ maior) $\Rightarrow$ espectro mais \textbf{largo}
\item Frequência de corte de 3\,dB ocorre em $\omega = a$
\end{itemize}

\end{frame}

% ============================================

\begin{frame}{Exemplo 3: Pulso Exponencial Bilateral}

Considere um pulso exponencial bilateral (simétrico):

\[
f(t) = e^{-a|t|}, \quad a > 0
\]

Este sinal pode ser decomposto em:
\[
f(t) = e^{-at}u(t) + e^{at}u(-t) = e^{-at}u(t) + e^{-a(-t)}u(-t)
\]

\textbf{Cálculo:}

\[
F(\omega) = \int_{-\infty}^{0} e^{at} e^{-j\omega t} dt + \int_{0}^{\infty} e^{-at} e^{-j\omega t} dt
\]

Para $t < 0$: $\int_{-\infty}^{0} e^{(a-j\omega)t} dt = \frac{1}{a-j\omega}$

Para $t > 0$: $\int_{0}^{\infty} e^{-(a+j\omega)t} dt = \frac{1}{a+j\omega}$

\end{frame}

% ============================================

\begin{frame}{Exemplo 3: Resultado}

\[
F(\omega) = \frac{1}{a-j\omega} + \frac{1}{a+j\omega} = \frac{(a+j\omega) + (a-j\omega)}{(a-j\omega)(a+j\omega)}
\]

\[
= \frac{2a}{a^2+\omega^2}
\]

\[
\boxed{f(t) = e^{-a|t|} \quad \xleftrightarrow{\ft} \quad F(\omega) = \frac{2a}{a^2+\omega^2}}
\]

\textbf{Observações:}

\begin{itemize}
\item $F(\omega)$ é \textbf{real} (fase zero) porque $f(t)$ é par
\item Espectro Lorentziano centrado em $\omega = 0$
\item $F(0) = 2/a$, $F(\pm a) = 1/a$
\item Forma mais concentrada no tempo que o exponencial unilateral
\end{itemize}

\end{frame}

% ============================================

\subsection{Interpretação Física}

\begin{frame}{Interpretação Física da Transformada de Fourier}

A transformada pode ser vista como uma \textbf{correlação} do sinal com exponenciais complexas:

\[
F(\omega) = \int_{-\infty}^{\infty} f(t) e^{-j\omega t} dt
\]

\begin{itemize}
\item $e^{-j\omega t} = \cos(\omega t) - j\sin(\omega t)$ é uma oscilação na frequência $\omega$
\item O produto $f(t) e^{-j\omega t}$ mede quanto de $f(t)$ "combina" com essa frequência
\item A integral acumula essa correlação sobre todo o tempo
\item $|F(\omega)|$ grande $\rightarrow$ forte presença da frequência $\omega$ em $f(t)$
\item $|F(\omega)|$ pequeno $\rightarrow$ pouca presença da frequência $\omega$
\end{itemize}

\vspace{0.3cm}

\textbf{Analogia:} A transformada de Fourier "decompõe" o sinal em componentes de frequência, assim como um prisma decompõe luz branca em cores.

\end{frame}

% ============================================

\begin{frame}{Princípio da Incerteza Tempo-Frequência}

Dos exemplos, observamos uma relação fundamental:

\begin{center}
\textbf{Duração no tempo} $\times$ \textbf{Largura de banda} $\approx$ constante
\end{center}

\vspace{0.3cm}

Matematicamente (Desigualdade de Heisenberg-Gabor):

\[
\Delta t \cdot \Delta \omega \geq \frac{1}{2}
\]

onde $\Delta t$ é a duração efetiva e $\Delta \omega$ é a largura de banda efetiva.

\vspace{0.3cm}

\textbf{Consequências práticas:}

\begin{itemize}
\item Pulsos curtos requerem grande largura de banda
\item Sinais de banda estreita devem ter longa duração
\item Compromisso fundamental em sistemas de comunicação
\end{itemize}

\end{frame}

% ============================================

\begin{frame}{Resumo da Seção 3.1}

\textbf{Conceitos fundamentais:}

\begin{itemize}
\item Transformada de Fourier: generalização da série de Fourier para sinais não-periódicos
\item Par de transformadas: $f(t) \xleftrightarrow{\ft} F(\omega)$
\item Espectro: magnitude e fase descrevem conteúdo em frequência
\end{itemize}

\vspace{0.3cm}

\textbf{Pares importantes derivados:}

\begin{itemize}
\item $\rect(t/\tau) \xleftrightarrow{\ft} \tau\sinc(\omega\tau/2\pi)$
\item $e^{-at}u(t) \xleftrightarrow{\ft} 1/(a+j\omega)$
\item $e^{-a|t|} \xleftrightarrow{\ft} 2a/(a^2+\omega^2)$
\end{itemize}

\vspace{0.3cm}

\textbf{Princípio fundamental:}
\begin{center}
Sinal concentrado no tempo $\leftrightarrow$ Espectro disperso em frequência
\end{center}

\end{frame}
