% ============================================
% SEÇÃO 6.2: CODIFICAÇÃO DE LINHA
% Principais formatos e suas propriedades.
% ============================================

% ------------------------------------------------------------------
\subsection{O que é Codificação de Linha?}
% ------------------------------------------------------------------

\begin{frame}{Onde Estamos na Cadeia?}
  \drawpipeline{linecoder}
  \vspace{0.1cm}
  {\small Nesta seção: \textbf{Codificador de Linha} — como converter
  bits em níveis de tensão, escolhendo o formato adequado para
  banda, sincronismo e componente DC.}
\end{frame}

\begin{frame}{Motivação}
  Antes de transmitir, precisamos escolher \textbf{como representar}
  cada bit (ou grupo de bits) como uma forma de onda elétrica.

  \vspace{0.3cm}
  A \alert{codificação de linha} define o mapeamento:
  \[
    \text{bit } b_k \;\longrightarrow\; \text{nível de tensão } a_k
    \;\longrightarrow\; \text{pulso } a_k\, p(t - kT_b)
  \]

  \vspace{0.3cm}
  \textbf{Critérios de escolha:}
  \begin{itemize}
    \item \textbf{Largura de banda} — quanto espectro o código ocupa?
    \item \textbf{Componente DC} — zero é desejável (acoplamento AC)
    \item \textbf{Sincronismo} — o código facilita recuperação de relógio?
    \item \textbf{Detecção de erros} — violações indicam erros?
    \item \textbf{Complexidade} — custo de implementação
  \end{itemize}
\end{frame}

% ------------------------------------------------------------------
\subsection{Códigos Clássicos}
% ------------------------------------------------------------------

\begin{frame}{Unipolar NRZ e Polar NRZ}
  \textbf{Unipolar NRZ} (\textit{Non-Return-to-Zero}):
  \[
    a_k = \begin{cases} A, & b_k = 1 \\ 0, & b_k = 0 \end{cases}
  \]
  \begin{itemize}
    \item Simples, mas tem \alert{componente DC} — ineficiente.
    \item Longas sequências de 0s → perda de sincronismo.
  \end{itemize}

  \vspace{0.4cm}
  \textbf{Polar NRZ:}
  \[
    a_k = \begin{cases} +A, & b_k = 1 \\ -A, & b_k = 0 \end{cases}
  \]
  \begin{itemize}
    \item Sem DC \textit{na média} (bits equiprováveis).
    \item Primeiro nulo do espectro em $f = R_b = 1/T_b$.
    \item Ainda tem problema de sincronismo em sequências longas.
  \end{itemize}
\end{frame}

% ------------------------------------------------------------------

\begin{frame}{Polar RZ e Manchester}
  \textbf{Polar RZ} (\textit{Return-to-Zero}):
  \[
    a_k = \begin{cases} \pm A & \text{na 1ª metade de } T_b \\
    0 & \text{na 2ª metade de } T_b \end{cases}
  \]
  \begin{itemize}
    \item Transição garantida a cada bit → melhor para sincronismo.
    \item Porém, ocupa o \alert{dobro da banda} do NRZ.
  \end{itemize}

  \vspace{0.4cm}
  \textbf{Manchester} (Bifase):
  \[
    b_k = 1 \;\Rightarrow\; +A \text{ (1ª metade)},\; -A \text{ (2ª metade)}
  \]
  \[
    b_k = 0 \;\Rightarrow\; -A \text{ (1ª metade)},\; +A \text{ (2ª metade)}
  \]
  \begin{itemize}
    \item \alert{Sem componente DC} — transição no meio de cada bit.
    \item Usado em Ethernet 10BASE-T.
    \item Largura de banda: primeiro nulo em $f = 2R_b$.
  \end{itemize}
\end{frame}

% ------------------------------------------------------------------

\begin{frame}{AMI — Inversão de Marca Alternada}
  No código \textbf{AMI} (\textit{Alternate Mark Inversion}):
  \[
    b_k = 0 \;\Rightarrow\; a_k = 0 \qquad
    b_k = 1 \;\Rightarrow\; a_k = +A \text{ ou } -A \;\text{(alternado)}
  \]

  \vspace{0.3cm}
  \textit{Cada ``1'' alterna a polaridade em relação ao ``1'' anterior.}

  \vspace{0.3cm}
  \textbf{Vantagens:}
  \begin{itemize}
    \item \alert{Componente DC = 0} por construção.
    \item Violação da regra de alternância → \textbf{detecção de erros}.
    \item PSD concentrada em frequências mais baixas.
  \end{itemize}

  \vspace{0.3cm}
  \textbf{Desvantagem:} longas sequências de ``0'' causam perda de
  sincronismo (o sinal fica em zero). \\
  \textit{Solução: usar \textbf{scrambling} (embaralhamento) ou
  códigos derivados como HDB3 e B8ZS.}
\end{frame}

% ------------------------------------------------------------------
\subsection{Formas de Onda e Espectros}
% ------------------------------------------------------------------

\begin{frame}{Comparação Visual das Codificações}
  \begin{center}
    \includegraphics[width=0.78\textwidth]{figures/cap6/line_coding_waveforms}
  \end{center}
  \vspace{-0.2cm}
  {\footnotesize
    Sequência de bits $[1,0,1,1,0,0,1,0]$ codificada em seis formatos diferentes.
    Note as transições e os níveis de tensão de cada código.}
\end{frame}

% ------------------------------------------------------------------

\begin{frame}{Densidade Espectral de Potência}
  \begin{center}
    \includegraphics[width=0.82\textwidth]{figures/cap6/line_coding_psd}
  \end{center}
  \vspace{-0.2cm}
  {\footnotesize
    A PSD do \textbf{Polar NRZ} tem primeiro nulo em $f = R_b$;
    \textbf{Manchester} tem nulo em $f = 2R_b$ (dobro da banda);
    \textbf{AMI} vai a zero em $f = 0$ (sem DC) e em $f = R_b$.}
\end{frame}

% ------------------------------------------------------------------
\subsection{Resumo Comparativo}
% ------------------------------------------------------------------

\begin{frame}{Tabela Comparativa dos Códigos de Linha}
  \vspace{0.2cm}
  \begin{center}
  \renewcommand{\arraystretch}{1.3}
  \footnotesize
  \begin{tabular}{l c c c c}
    \hline
    \textbf{Código} & \textbf{BW mín.} & \textbf{DC = 0?} &
    \textbf{Sincronismo} & \textbf{Detecção} \\
    \hline
    Unipolar NRZ   & $R_b$    & Não  & Ruim   & Não \\
    Polar NRZ       & $R_b$    & Sim* & Ruim   & Não \\
    Polar RZ        & $2R_b$   & Sim* & Bom    & Não \\
    Manchester      & $2R_b$   & Sim  & Ótimo  & Não \\
    AMI (NRZ)       & $R_b$    & Sim  & Médio  & Sim \\
    Bipolar RZ      & $R_b$    & Sim  & Bom    & Sim \\
    \hline
  \end{tabular}
  \end{center}
  \vspace{0.2cm}
  {\footnotesize *Na média, para bits equiprováveis.
  \textbf{BW mín.} = largura de banda do primeiro lóbulo espectral.}

  \vspace{0.2cm}
  \textit{Não existe código perfeito — a escolha depende dos requisitos
  do sistema (banda disponível, necessidade de sincronismo, etc.).}
\end{frame}
