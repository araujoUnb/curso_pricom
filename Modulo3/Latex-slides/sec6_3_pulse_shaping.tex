% ============================================
% SEÇÃO 6.3: FORMATAÇÃO DE PULSO E ISI
% Critério de Nyquist, pulso cosseno levantado.
% ============================================

% ------------------------------------------------------------------
\subsection{Interferência Intersimbólica (ISI)}
% ------------------------------------------------------------------

\begin{frame}{Onde Estamos na Cadeia?}
  \drawpipeline{pulseshaper}
  \vspace{0.1cm}
  {\small Nesta seção: \textbf{Formatador de Pulso} — a escolha do pulso
  $p(t)$ determina a banda ocupada e a sensibilidade à ISI.
  O critério de Nyquist e o cosseno levantado são as ferramentas centrais.}
\end{frame}

\begin{frame}{O Problema da ISI}
  Considere o sinal transmitido:
  \[
    s(t) = \sum_{k} a_k\, p(t - kT)
  \]

  No receptor, amostramos $r(t)$ nos instantes $t = nT$:
  \[
    r(nT) = \underbrace{a_n\, p(0)}_{\text{símbolo desejado}}
    + \underbrace{\sum_{k \neq n} a_k\, p(nT - kT)}_{\text{ISI}}
    + n(nT)
  \]

  \vspace{0.3cm}
  Se $p(t)$ for ``largo demais'', as \alert{caudas} de pulsos vizinhos
  contaminam o instante de amostragem → \textbf{Interferência Intersimbólica (ISI)}.

  \vspace{0.3cm}
  \textit{A ISI é tão prejudicial quanto o ruído — e pode ser eliminada
  com a escolha adequada de $p(t)$.}
\end{frame}

% ------------------------------------------------------------------

\begin{frame}{Visualização da ISI}
  \begin{center}
    \includegraphics[width=0.78\textwidth]{figures/cap6/pulse_shaping_isi}
  \end{center}
  \vspace{-0.2cm}
  {\footnotesize
    (a) Pulsos sinc ideais têm zeros em $t = nT$ ($n \neq 0$) → sem ISI.
    (b) Pulsos mais ``gordos'' geram sobreposição → ISI nos instantes de decisão.}
\end{frame}

% ------------------------------------------------------------------
\subsection{Critério de Nyquist para Zero ISI}
% ------------------------------------------------------------------

\begin{frame}{Condição para ISI Nula}
  Para eliminar a ISI, precisamos que $p(t)$ satisfaça:
  \[
    p(nT) = \begin{cases} 1, & n = 0 \\ 0, & n \neq 0 \end{cases}
  \]

  \vspace{0.2cm}
  Isso equivale, no domínio da frequência, ao \alert{Critério de Nyquist}:
  \[
    \sum_{k=-\infty}^{\infty} P\!\left(f - \frac{k}{T}\right) = T
    \quad \forall\, f
  \]

  \vspace{0.3cm}
  \textit{Interpretação:} quando ``dobramos'' (somamos réplicas de) $P(f)$
  com espaçamento $1/T$, o resultado deve ser \textbf{constante}.

  \vspace{0.2cm}
  O pulso mais simples que satisfaz isso é o $\sinc$:
  \[
    p(t) = \sinc\!\left(\frac{t}{T}\right) \quad \Longleftrightarrow \quad
    P(f) = T\,\rect(fT)
  \]
  Banda mínima: $W = \dfrac{1}{2T}$ (filtro retangular ideal).
\end{frame}

% ------------------------------------------------------------------

\begin{frame}{Problema do Pulso Sinc Puro}
  O pulso $\sinc(t/T)$ é \textbf{matematicamente perfeito}, mas na prática:

  \vspace{0.3cm}
  \begin{itemize}
    \item \textbf{Decaimento lento} — caudas decaem como $1/t$ apenas.
    \item Pequeno \alert{erro de temporização} $\epsilon$ causa ISI severa
          (a série $\sum 1/n$ diverge!).
    \item \textbf{Não é realizável} — resposta ao impulso infinita e não causal.
  \end{itemize}

  \vspace{0.4cm}
  \textit{Solução prática:} usar um pulso com \textbf{banda extra}
  (roll-off) que ainda satisfaz o critério de Nyquist, mas cujas
  caudas decaem mais rápido → \alert{Cosseno Levantado}.
\end{frame}

% ------------------------------------------------------------------
\subsection{Pulso Cosseno Levantado}
% ------------------------------------------------------------------

\begin{frame}{Definição — Domínio da Frequência}
  O espectro do \alert{cosseno levantado} com fator de
  \textit{roll-off} $\alpha \in [0, 1]$:
  \[
    P(f) = \begin{cases}
      T, & |f| \le \dfrac{1-\alpha}{2T} \\[8pt]
      \dfrac{T}{2}\left[1 + \cos\!\left(
        \dfrac{\pi T}{\alpha}\left(|f| - \dfrac{1-\alpha}{2T}\right)
      \right)\right], & \dfrac{1-\alpha}{2T} < |f| \le \dfrac{1+\alpha}{2T} \\[8pt]
      0, & |f| > \dfrac{1+\alpha}{2T}
    \end{cases}
  \]

  \vspace{0.2cm}
  A \textbf{largura de banda} ocupada é:
  \[
    W = \frac{1+\alpha}{2T}
  \]

  \vspace{0.1cm}
  {\footnotesize
  $\alpha = 0$: banda mínima ($W = 1/2T$), mas caudas lentas (sinc puro). \\
  $\alpha = 1$: dobro da banda ($W = 1/T$), mas caudas decaem como $1/t^3$ — muito mais robusto.}
\end{frame}

% ------------------------------------------------------------------

\begin{frame}{Definição — Domínio do Tempo}
  No tempo, o pulso cosseno levantado é:
  \[
    p(t) = \sinc\!\left(\frac{t}{T}\right) \cdot
    \frac{\cos\!\left(\dfrac{\pi \alpha\, t}{T}\right)}
    {1 - \left(\dfrac{2\alpha\, t}{T}\right)^{\!2}}
  \]

  \vspace{0.3cm}
  \textbf{Observação importante:}
  \begin{itemize}
    \item O fator $\sinc(t/T)$ garante zeros em $t = nT$ → critério de Nyquist.
    \item O fator cosseno/denominador acelera o decaimento das caudas.
    \item Para $\alpha > 0$, as caudas decaem como $\sim 1/|t|^3$.
  \end{itemize}

  \vspace{0.2cm}
  \textit{Quanto maior $\alpha$, mais robusto ao erro de temporização,
  porém maior a banda ocupada.}
\end{frame}

% ------------------------------------------------------------------
\subsection{Visualização}
% ------------------------------------------------------------------

\begin{frame}{Cosseno Levantado: Tempo e Frequência}
  \begin{columns}[T]
    \column{0.49\textwidth}
    \textbf{Domínio do tempo}
    \begin{center}
      \includegraphics[width=\textwidth]{figures/cap6/raised_cosine_time}
    \end{center}
    {\scriptsize
      Todos os pulsos passam por zero em $t = nT$.
      Com $\alpha = 1$, as caudas decaem como $1/|t|^3$.}

    \column{0.49\textwidth}
    \textbf{Domínio da frequência}
    \begin{center}
      \includegraphics[width=\textwidth]{figures/cap6/raised_cosine_freq}
    \end{center}
    {\scriptsize
      $\alpha = 0$: filtro retangular (banda mínima $1/2T$).
      $\alpha = 1$: transição suave, banda dobrada $1/T$.}
  \end{columns}
\end{frame}

% ------------------------------------------------------------------

\begin{frame}{Comparação: Retangular vs.\ Sinc vs.\ Cosseno Levantado}
  \begin{center}
    \includegraphics[width=0.88\textwidth]{figures/cap6/pulse_shaping_comparison}
  \end{center}
  \vspace{-0.2cm}
  {\footnotesize
    O sinal total resulta da soma de pulsos deslocados. Em (a), as
    descontinuidades causam banda infinita. Em (b) e (c), a suavização
    limita o espectro, mas apenas (b) e (c) eliminam ISI nos instantes $nT$.}
\end{frame}

% ------------------------------------------------------------------
\subsection{Eficiência Espectral}
% ------------------------------------------------------------------

\begin{frame}{Relação Banda--Taxa}
  Com pulso cosseno levantado de roll-off $\alpha$, a banda necessária é:
  \[
    W = \frac{1 + \alpha}{2T} = \frac{(1+\alpha)\, R_s}{2}
  \]

  \vspace{0.2cm}
  A \textbf{eficiência espectral} (para sistema binário) é:
  \[
    \eta = \frac{R_b}{W} = \frac{2}{1 + \alpha} \quad
    \text{[bits/s/Hz]}
  \]

  \vspace{0.3cm}
  \begin{center}
  \renewcommand{\arraystretch}{1.2}
  \footnotesize
  \begin{tabular}{c c c}
    \hline
    $\alpha$ & Banda $W$ & Eficiência $\eta$ \\
    \hline
    0    & $R_b/2$  & 2 bits/s/Hz \\
    0.5  & $0{,}75\, R_b$ & 1{,}33 bits/s/Hz \\
    1.0  & $R_b$    & 1 bit/s/Hz \\
    \hline
  \end{tabular}
  \end{center}

  \vspace{0.2cm}
  \textit{Na prática, $\alpha$ entre 0,2 e 0,5 oferece bom
  compromisso entre banda e robustez.}
\end{frame}
