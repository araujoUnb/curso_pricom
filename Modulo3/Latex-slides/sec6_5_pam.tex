% ============================================
% SEÇÃO 6.5: PAM — SINALIZAÇÃO M-ÁRIA
% Extensão para múltiplos níveis.
% ============================================

% ------------------------------------------------------------------
\subsection{Motivação para M-PAM}
% ------------------------------------------------------------------

\begin{frame}{Onde Estamos na Cadeia?}
  \drawpipeline{pulseshaper,detector}
  \vspace{0.1cm}
  {\small Nesta seção: \textbf{$M$-PAM} — ampliamos o alfabeto de
  símbolos para $M$ níveis, aumentando a eficiência espectral ao
  custo de maior sensibilidade ao ruído.}
\end{frame}

\begin{frame}{Por que Usar Mais de 2 Níveis?}
  No sistema binário (2-PAM), cada símbolo carrega \textbf{1 bit}.

  \vspace{0.3cm}
  Se usarmos $M$ níveis de amplitude (\alert{$M$-PAM}), cada símbolo
  carrega $\log_2 M$ bits:
  \[
    R_b = R_s \cdot \log_2 M
  \]

  \vspace{0.2cm}
  Para uma mesma banda $W$, podemos \textbf{aumentar a taxa de bits}
  sem aumentar a taxa de símbolos $R_s$ — ou, equivalentemente,
  manter $R_b$ e \alert{reduzir pela metade} a banda a cada
  duplicação de $M$.

  \vspace{0.3cm}
  \textit{Exemplo prático:} Ethernet 2.5GBASE-T e 5GBASE-T usam
  \textbf{PAM-4} e \textbf{PAM-16} para atingir altas taxas em
  cabos de par trançado.
\end{frame}

% ------------------------------------------------------------------
\subsection{Definição do Sinal M-PAM}
% ------------------------------------------------------------------

\begin{frame}{Sinal $M$-PAM}
  O sinal transmitido é:
  \[
    s(t) = \sum_{k} a_k\, p(t - kT)
  \]
  onde agora $a_k \in \{-(M-1),\; -(M-3),\; \ldots,\; +(M-3),\; +(M-1)\}$.

  \vspace{0.3cm}
  Os $M$ níveis são \textbf{igualmente espaçados} com distância $2d$:
  \[
    a_k \in \{(2m - 1 - M)\, d \;:\; m = 1, 2, \ldots, M\}
  \]

  \vspace{0.3cm}
  \textit{Exemplo para 4-PAM} ($M = 4$, $d = 1$):
  \[
    a_k \in \{-3,\; -1,\; +1,\; +3\}
  \]
  Cada símbolo representa $\log_2 4 = 2$ bits:
  $00 \to -3$, $01 \to -1$, $10 \to +1$, $11 \to +3$.
\end{frame}

% ------------------------------------------------------------------
\subsection{Constelação e Formas de Onda}
% ------------------------------------------------------------------

\begin{frame}{Constelações PAM}
  \begin{center}
    \includegraphics[width=0.78\textwidth]{figures/cap6/pam_constellation}
  \end{center}
  \vspace{-0.2cm}
  {\footnotesize
    A ``constelação'' de um sinal $M$-PAM é um conjunto de pontos
    sobre uma linha real. As linhas tracejadas indicam os limiares de
    decisão. Com mais níveis, a distância entre vizinhos diminui.}
\end{frame}

% ------------------------------------------------------------------

\begin{frame}{Comparação de Formas de Onda: 2-PAM vs.\ 4-PAM}
  \begin{center}
    \includegraphics[width=0.78\textwidth]{figures/cap6/pam_waveforms}
  \end{center}
  \vspace{-0.2cm}
  {\footnotesize
    Em (a), cada bit gera um símbolo — taxa de símbolo igual à de bits.
    Em (b), dois bits são agrupados em um símbolo 4-PAM — a taxa de
    símbolo cai pela metade, economizando banda.}
\end{frame}

% ------------------------------------------------------------------
\subsection{Probabilidade de Erro}
% ------------------------------------------------------------------

\begin{frame}{Decisão e Probabilidade de Erro}
  No receptor, o sinal amostrado é comparado com \textbf{limiares}:
  \[
    \hat{a}_k = \arg\min_{a_m} |r(kT) - a_m|
  \]

  \vspace{0.3cm}
  Para $M$-PAM com ruído gaussiano $\sigma^2$ e distância $2d$:

  \vspace{0.2cm}
  A probabilidade de erro de \textbf{símbolo} é:
  \[
    P_s = 2\left(1 - \frac{1}{M}\right) Q\!\left(\frac{d}{\sigma}\right)
  \]

  \vspace{0.2cm}
  onde $Q(x) = \frac{1}{\sqrt{2\pi}} \int_x^{\infty} e^{-u^2/2}\, du$.

  \vspace{0.3cm}
  \textit{Observação:} quanto maior $M$, menor a distância $d$
  (para mesma potência média), e portanto \alert{maior a taxa de erro}.
  Há um compromisso entre eficiência espectral e desempenho.
\end{frame}

% ------------------------------------------------------------------

\begin{frame}{Energia Média e SNR}
  A \textbf{energia média por símbolo} do $M$-PAM é:
  \[
    E_s = \frac{(M^2 - 1)\, d^2}{3}
  \]

  Substituindo $d = \sqrt{3 E_s / (M^2 - 1)}$ na expressão de $P_s$:
  \[
    P_s = 2\left(1 - \frac{1}{M}\right)
    Q\!\left(\sqrt{\frac{3\, E_s}{(M^2 - 1)\,\sigma^2}}\right)
  \]

  \vspace{0.3cm}
  Em termos da \textbf{energia por bit} $E_b = E_s / \log_2 M$:
  \[
    P_s = 2\left(1 - \frac{1}{M}\right)
    Q\!\left(\sqrt{\frac{3\, \log_2 M}{M^2 - 1} \cdot
    \frac{2 E_b}{N_0}}\right)
  \]

  \vspace{0.2cm}
  \textit{Aumentar $M$ exige maior $E_b/N_0$ para manter a mesma
  $P_s$ — é o ``custo'' da eficiência espectral.}
\end{frame}

% ------------------------------------------------------------------
\subsection{Diagrama de Olho M-PAM}
% ------------------------------------------------------------------

\begin{frame}{Diagrama de Olho para 4-PAM}
  \begin{center}
    \includegraphics[width=0.78\textwidth]{figures/cap6/eye_diagram_4pam}
  \end{center}
  \vspace{-0.2cm}
  {\footnotesize
    Com 4 níveis, aparecem 3 ``olhos'' empilhados. As linhas tracejadas
    amarelas indicam os limiares de decisão. O olho central é menor que
    no caso binário — \alert{maior sensibilidade ao ruído}.}
\end{frame}

% ------------------------------------------------------------------
\subsection{Resumo do Capítulo}
% ------------------------------------------------------------------

\begin{frame}{Resumo — Transmissão Digital em Banda Base}
  \textbf{Codificação de Linha:} define como bits viram formas de onda.
  Códigos como Polar NRZ, Manchester e AMI têm diferentes compromissos
  em banda, DC e sincronismo.

  \vspace{0.3cm}
  \textbf{Formatação de Pulso:} o pulso $p(t)$ deve satisfazer o
  critério de Nyquist para eliminar ISI. O cosseno levantado com
  roll-off $\alpha$ é a escolha padrão:
  \[
    W = \frac{(1+\alpha)\, R_s}{2}
  \]

  \vspace{0.3cm}
  \textbf{Diagrama de Olho:} ferramenta visual que revela margem de
  ruído, margem de temporização e quantidade de ISI.

  \vspace{0.3cm}
  \textbf{$M$-PAM:} usando $M$ níveis, a eficiência espectral aumenta
  ($\log_2 M$ bits/símbolo), mas a sensibilidade ao ruído cresce.
\end{frame}
