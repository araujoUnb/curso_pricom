\documentclass[11pt,a4paper]{article}
\usepackage[utf8]{inputenc}
\usepackage[T1]{fontenc}
\usepackage[brazilian]{babel}
\usepackage{geometry}
\usepackage{enumitem}
\usepackage{booktabs}
\usepackage{amsmath}
\usepackage{graphicx}
\graphicspath{{../figures_global/}}
\usepackage{fancyhdr}
\usepackage[backend=biber, style=numeric, sorting=none]{biblatex}
\usepackage{hyperref}
\addbibresource{referencias.bib}

\geometry{margin=2.5cm, headheight=32pt}
\setlength{\parindent}{0pt}
\setlength{\parskip}{0.5em}

\pagestyle{fancy}
\fancyhf{}
\fancyhead[L]{\small\textbf{Faculdade de Ciências e Tecnologias em Engenharia}}
\fancyhead[R]{\includegraphics[height=0.85cm]{../figures_global/unb.png}}
\renewcommand{\headrulewidth}{0.4pt}
\fancyfoot[C]{\small FGA0092 -- Princípios de Comunicação para Engenharia}
\renewcommand{\footrulewidth}{0.4pt}
\fancypagestyle{plain}{%
  \fancyhf{}%
  \fancyhead[L]{\small\textbf{Faculdade de Ciências e Tecnologias em Engenharia}}%
  \fancyhead[R]{\includegraphics[height=0.85cm]{../figures_global/unb.png}}%
  \renewcommand{\headrulewidth}{0.4pt}%
  \fancyfoot[C]{\small FGA0092 -- Princípios de Comunicação para Engenharia}
  \renewcommand{\footrulewidth}{0.4pt}%
}

\title{\textbf{Plano de Ensino}}
\author{FGA0092 -- Princípios de Comunicação para Engenharia}
\date{}

\begin{document}

\thispagestyle{plain}
\maketitle
\vspace{-1.2em}


\vspace{1em}

\section{Descrição Geral}

Esta disciplina aborda os fundamentos de sistemas de comunicação modernos, com ênfase em análise espectral, técnicas de modulação analógica e digital, ruído em sistemas de comunicação e noções de protocolos e redes. Serão estudados: espectro e sinais (Série e Transformada de Fourier, densidade espectral); modulações analógicas (AM, FM, PM, PLL); ruído em sistemas analógicos; comunicação digital em banda-base (amostragem, quantização, PCM); desempenho de enlaces digitais (taxa de erro de bit); e noções de protocolos e redes. O curso fornece base para disciplinas avançadas de processamento de sinais, comunicações digitais e sistemas de telecomunicações.

\section{Pré-requisitos}

\begin{itemize}[noitemsep]
  \item FGA0102 -- Sinais e Sistemas
  \item FGA0067 -- Teoria de Circuitos Eletrônicos I
\end{itemize}

\section{Professor}

\textbf{Nome:} Daniel C. Araújo

O contato pode ser feito via e-mail (\href{mailto:daniel.araujo@unb.br}{daniel.araujo@unb.br}), pelo Microsoft Teams ou pelo SIGAA.

\section{As aulas}

O curso será ministrado presencialmente no campus da FCTE. O material da disciplina será disponibilizado pelo SIGAA.

\section{Objetivos}

\begin{itemize}[noitemsep]
  \item Compreender a análise espectral de sinais e sua aplicação em sistemas de comunicação.
  \item Dominar as técnicas de modulação analógica (AM, FM, PM) e suas variantes.
  \item Analisar o efeito do ruído em sistemas analógicos e digitais de comunicação.
  \item Entender os processos de amostragem, quantização e modulação por código de pulso (PCM).
  \item Avaliar o desempenho de enlaces de comunicação digital em termos de taxa de erro de bit.
  \item Obter noções básicas de protocolos e arquiteturas de redes de comunicação.
\end{itemize}

\section{Conteúdo Programático}

\begin{center}
\small
\begin{tabular}{|p{0.22\textwidth}|p{0.22\textwidth}|p{0.22\textwidth}|p{0.22\textwidth}|}
\hline
\textbf{Módulo 1: Espectro e Sinais} & \textbf{Módulo 2: Modulações Analógicas} & \textbf{Módulo 3: ADC e Digital} & \textbf{Lab PRICOM} \\
\hline
\parbox[t]{0.20\textwidth}{%
  \vspace{0.2em}
  \begin{itemize}[noitemsep, leftmargin=*, nosep, font=\small]
    \item Classificação de sinais
    \item Série de Fourier
    \item Transformada de Fourier
    \item Densidade Espectral de Energia
    \item Densidade Espectral de Potência
    \item Sinais passa-bandas
  \end{itemize}
  \vspace{0.2em}
} &
\parbox[t]{0.20\textwidth}{%
  \vspace{0.2em}
  \begin{itemize}[noitemsep, leftmargin=*, nosep, font=\small]
    \item Modulação em Amplitude (AM)
    \item Variantes: DSB, DSB+C, SSB, VSB
    \item Modulação em Frequência (FM)
    \item Modulação em Fase (PM)
    \item PLL
    \item Ruído em Sistemas Analógicos
    \item RSR em Demoduladores
  \end{itemize}
  \vspace{0.2em}
} &
\parbox[t]{0.20\textwidth}{%
  \vspace{0.2em}
  \begin{itemize}[noitemsep, leftmargin=*, nosep, font=\small]
    \item Amostragem
    \item Quantização
    \item PCM Linear e Não-linear
    \item Codificação de Linha
    \item Interferência Entre Símbolos (ISI)
    \item Taxa de Erro de Bit (TEB)
    \item Noções de Protocolos
  \end{itemize}
  \vspace{0.2em}
} &
\parbox[t]{0.20\textwidth}{%
  \vspace{0.2em}
  \begin{itemize}[noitemsep, leftmargin=*, nosep, font=\small]
    \item Análise Espectral (FFT)
    \item Modulação AM
    \item Modulação FM
    \item PLL
    \item Amostragem e Aliasing
    \item Quantização e PCM
    \item Análise de TEB
  \end{itemize}
  \vspace{0.2em}
} \\
\hline
\end{tabular}
\end{center}

\section{Avaliação}

A avaliação da disciplina é composta por duas componentes: \textbf{Teoria} e \textbf{Prática de Laboratório}.

\subsection*{Componente Teórica}

A nota teórica ($N_T$) é calculada a partir da média ponderada das três provas:

\[
N_T = \frac{P_1 + P_2 + 2 \cdot P_3}{4}
\]

\subsection*{Componente Prática}

A nota prática ($N_L$) é calculada a partir da média aritmética dos relatórios das oito práticas de laboratório:

\[
N_L = \frac{\text{Rel}_1 + \text{Rel}_2 + \cdots + \text{Rel}_8}{8}
\]

\subsection*{Nota Final e Aprovação}

A nota final ($N_F$) é a média ponderada das componentes teórica e prática:

\[
N_F = 0.75 \cdot N_T + 0.25 \cdot N_L
\]

\textbf{Condições para aprovação:}
\begin{itemize}[noitemsep]
  \item \underline{O aluno DEVE ser aprovado tanto na teoria quanto na prática}. Isso significa:
    \begin{itemize}[noitemsep, leftmargin=1cm]
      \item $N_T \geq 5$ (nota na teoria)
      \item $N_L \geq 5$ (nota na prática)
    \end{itemize}
  \item \underline{É obrigatória a participação na última prova teórica (P3)}. Caso o aluno não esteja presente ou tire zero, será automaticamente reprovado.
  \item A aprovação final é obtida se $N_F \geq 5$.
  \item A frequência mínima de 75\% nas aulas teóricas é obrigatória.
  \item A presença mínima de 75\% na prática de laboratório é obrigatória.
\end{itemize}



\section{Referências}

\nocite{*}
\printbibliography[heading=none]

\end{document}
