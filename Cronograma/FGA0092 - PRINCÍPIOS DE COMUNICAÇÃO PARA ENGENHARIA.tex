\documentclass[11pt,a4paper]{article}
\usepackage[utf8]{inputenc}
\usepackage[T1]{fontenc}
\usepackage[brazilian]{babel}
\usepackage{geometry}
\usepackage{enumitem}
\usepackage{booktabs}
\usepackage{graphicx}
\graphicspath{{../figures_global/}}
\usepackage{fancyhdr}
\usepackage{hyperref}
\usepackage{array}
\usepackage{longtable}

\geometry{margin=2cm, headheight=32pt}
\setlength{\parindent}{0pt}
\setlength{\parskip}{0.5em}

\pagestyle{fancy}
\fancyhf{}
\fancyhead[L]{\small\textbf{Faculdade de Ciências e Tecnologias em Engenharia}}
\fancyhead[R]{\includegraphics[height=0.85cm]{../figures_global/unb.png}}
\renewcommand{\headrulewidth}{0.4pt}
\fancypagestyle{plain}{%
  \fancyhf{}%
  \fancyhead[L]{\small\textbf{Faculdade de Ciências e Tecnologias em Engenharia}}%
  \fancyhead[R]{\includegraphics[height=0.85cm]{../figures_global/unb.png}}%
  \renewcommand{\headrulewidth}{0.4pt}%
}

\title{\textbf{Plano de Ensino e Cronograma}}
\author{FGA0092 -- Princípios de Comunicação para Engenharia}
\date{}

\begin{document}

\thispagestyle{plain}
\maketitle
\vspace{-1.2em}

\begin{center}
\textbf{FGA0092 -- Princípios de Comunicação para Engenharia} \\
\textbf{Período: 16/03/2026 a 18/07/2026}
\end{center}
\vspace{0.5em}

\noindent\textbf{Marcos do calendário acadêmico:} Primeiro dia de aulas 16/03 -- 25\% de realização 15/04 -- 50\% 18/05 -- 75\% 19/06 -- Último dia de aulas 18/07.
\vspace{0.5em}

\noindent\textit{Este calendário é uma orientação e pode ser ajustado ao longo do semestre. As datas das provas poderão ser modificadas de acordo com as demandas da disciplina.}
\vspace{1em}

\section{Conteúdo Programático}

\begin{center}
\small
\begin{tabular}{|p{0.22\textwidth}|p{0.22\textwidth}|p{0.22\textwidth}|p{0.22\textwidth}|}
\hline
\textbf{Módulo 1: Espectro e Sinais} & \textbf{Módulo 2: Modulações Analógicas} & \textbf{Módulo 3: ADC e Comunicação Digital} & \textbf{Lab PRICOM} \\
\hline
\parbox[t]{0.20\textwidth}{%
  \vspace{0.2em}
  \begin{itemize}[noitemsep, leftmargin=*, nosep, font=\small]
    \item Classificação de sinais
    \item Série de Fourier
    \item Transformada de Fourier
    \item Densidade Espectral de Energia
    \item Densidade Espectral de Potência
    \item Sinais passa-bandas
  \end{itemize}
  \vspace{0.2em}
} &
\parbox[t]{0.20\textwidth}{%
  \vspace{0.2em}
  \begin{itemize}[noitemsep, leftmargin=*, nosep, font=\small]
    \item Modulação em Amplitude (AM)
    \item Variantes: DSB, DSB+C, SSB, VSB
    \item Modulação em Frequência (FM)
    \item Modulação em Fase (PM)
    \item PLL
    \item Ruído em Sistemas Analógicos
    \item RSR em Demoduladores
  \end{itemize}
  \vspace{0.2em}
} &
\parbox[t]{0.20\textwidth}{%
  \vspace{0.2em}
  \begin{itemize}[noitemsep, leftmargin=*, nosep, font=\small]
    \item Amostragem
    \item Quantização
    \item PCM Linear e Não-linear
    \item Codificação de Linha
    \item Interferência Entre Símbolos (ISI)
    \item Taxa de Erro de Bit (TEB)
    \item Noções de Protocolos e Redes
  \end{itemize}
  \vspace{0.2em}
} &
\parbox[t]{0.20\textwidth}{%
  \vspace{0.2em}
  \begin{itemize}[noitemsep, leftmargin=*, nosep, font=\small]
    \item Análise Espectral
    \item Modulação AM
    \item Modulação FM
    \item PLL
    \item Amostragem e Aliasing
    \item Quantização e PCM
    \item Análise de TEB
  \end{itemize}
  \vspace{0.2em}
} \\
\hline
\end{tabular}
\end{center}

\vspace{1.5em}

\section{Cronograma de Aulas}

\begin{center}
\small
\begin{longtable}{|c|c|p{5.0cm}|p{4.0cm}|p{1.8cm}|}
\hline
\textbf{Sem.} & \textbf{Datas} & \textbf{Conteúdo Teórico (T/Q)} & \textbf{Laboratório (Terça 15h)} & \textbf{Feriado} \\
\hline
\endfirsthead

\hline
\textbf{Sem.} & \textbf{Datas} & \textbf{Conteúdo Teórico (T/Q)} & \textbf{Laboratório (Terça 15h)} & \textbf{Feriado} \\
\hline
\endhead

\hline
\endfoot

1 & 16/03--20/03 & Apresentação. Classificação de sinais & Apresentação e orientações & --- \\
\hline
2 & 23/03--27/03 & Série de Fourier, Transformada de Fourier & Prática 1: FFT e Análise Espectral & --- \\
\hline
3 & 30/03--03/04 & Modulação AM (DSB, DSB+C, SSB, VSB) & Prática 1 (continuação) & Sexta Santa \\
\hline
4 & 06/04--10/04 & Densidade Espectral, Sinais passa-bandas & Prática 2: Modulação AM-DSB-SC & --- \\
\hline
5 & 13/04--17/04 & Fase instantânea, Conceitos fundamentais & Prática 2 (continuação) & --- \\
\hline
6 & 20/04--24/04 & \textbf{PROVA 1} -- Módulo 1 & (Sem aula -- Tiradentes 21/04) & Tiradentes \\
\hline
7 & 27/04--01/05 & Modulação em Frequência (FM) & Prática 3: Modulação AM Convencional & Dia Trab. \\
\hline
8 & 04/05--08/05 & Largura de banda FM, Índice de modulação & Prática 3 (continuação) & --- \\
\hline
9 & 11/05--15/05 & Modulação em Fase (PM) e suas características & Prática 4: Modulação SSB & --- \\
\hline
10 & 18/05--22/05 & Receptores superheteródinos & Prática 4 (continuação) & --- \\
\hline
11 & 25/05--29/05 & Ruído em sistemas analógicos & Prática 5: Modulação FM & --- \\
\hline
12 & 01/06--05/06 & RSR em demoduladores AM/FM & Prática 5 (continuação) & Corpus Chr. \\
\hline
13 & 08/06--12/06 & \textbf{PROVA 2} -- Módulo 2 & Prática 6: Phase-Locked Loop (PLL) & --- \\
\hline
14 & 15/06--19/06 & Amostragem (Teorema de Nyquist, aliasing) & Prática 6 (continuação) & --- \\
\hline
15 & 22/06--26/06 & Quantização, PCM Linear e não-linear & Prática 7: Ruído em Demoduladores & --- \\
\hline
16 & 29/06--03/07 & Codificação de Linha, Critério Nyquist & Prática 7 (continuação) & --- \\
\hline
17 & 06/07--10/07 & Taxa de Erro de Bit (TEB), Protocolos/Redes & Prática 8: Amostragem e Aliasing & --- \\
\hline
18 & 13/07--18/07 & \textbf{PROVA 3} -- Módulo 3 & Prática 8 (continuação) & --- \\
\hline

\end{longtable}
\end{center}

\vspace{0.5em}

\textbf{Obs. sobre Relatórios de Práticas:} Cada prática deve ter seu relatório entregue via SIGAA na semana seguinte ao término da prática. Por exemplo, a Prática 1 (semanas 2-3) tem relatório entregue até o fim da semana 4.

\vspace{0.5em}

\textit{Obs: T=Terça, Q=Quinta. Marcos: 15/04 (25\%), 18/05 (50\%), 19/06 (75\%), 18/07 (Último dia).}

\end{document}
